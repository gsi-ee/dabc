[programmer/prog-exa-mbs.tex]

\section{Overview}

MBS (Multi Branch System) is the standard DAQ system of GSI.
Support of MBS in \dabc~ includes several components:
\bbul
\item type definitions for different MBS structures  
\item iterator classes for reading/creating MBS event/subevent data 
\item support of new {\tt .lmd} file format  
\item \class{mbs::ClientTransport} for connecting to MBS servers  
\item \class{mbs::ServerTransport} to "emulate" running MBS servers  
\item \class{mbs::CombinerModule} for performing mbs events building   
\item \class{mbs::GeneratorModule} for generating random mbs events 
\ebul

This plugin is part of the standard \dabc~ distribution.  
All sources can be found in the {\tt \$DABCSYS/plugin/mbs} directory.
All these sources are compiled into library {\tt libDabcMbs.so}
which is located in {\tt \$DABCSYS/lib}. 

\section{Event iterators}

The MBS system has native event and subevent formats. 
To access such event data, several structures are introduced in 
{\tt mbs/LmdTypeDefs.h} and {\tt mbs/MbsTypeDefs.h}.
In the first file, structure \class{mbs::Header} is defined
which is just a container for arbitrary raw data. 
Such container is used to store data to, or to read data from 
{\tt .lmd} files, resp.
In file {\tt mbs/MbsTypeDefs.h} the following structures are defined:
\bbul
\item \class{mbs::EventHeader} - MBS event header of 10-1 type 
\item \class{mbs::SubeventHeader} - MBS subevent header of 10-1 type 
\ebul
\dabc~ operates with buffers of type \keyw{mbt\_MbsEvents} that contain
several subsequent MBS events; there is no buffer header in front here. 
To iterate over all events in such
buffer and to access them, class \class{mbs::ReadIterator} is provided 
(as defined in {\tt mbs/Iterator.h}).  
This is done in the following way:

\begin{small}
\begin{verbatim}
#include "mbs/Iterator.h"

void Print(dabc::Buffer* buf)
{
   mbs::ReadIterator iter(buf);
   while (iter.NextEvent()) {
      DOUT1(("Event %u size %u", 
              iter.evnt()->EventNumber(), 
              iter.evnt()->FullSize()));
      while (iter.NextSubEvent()) {
         DOUT1(("Subevent crate %u procid %u size %u",
               iter.subevnt()->iSubcrate, 
               iter.subevnt()->iProcId, 
               iter.subevnt()->FullSize()));
      }
   }
}
\end{verbatim}
\end{small}

Another class \class{mbs::WriteIterator} fills 
MBS events into \class{dabc::Buffer}. 
This is illustrated by the following code:

\begin{small}
\begin{verbatim}
#include "mbs/Iterator.h"

void Fill(dabc::Buffer* buf)
{
   mbs::WriteIterator iter(buf);
   unsigned evntid = 0;
   while (iter.NewEvent(evntid++)) {
      for (unsigned subcnt = 0; subcnt < 3; subcnt++) {
         if (!iter.NewSubevent(28, 0, subcnt)) return;
         // fill raw data iter.rawdata() here
         memset(iter.rawdata(), 0, 28);
         iter.FinishSubEvent(28);
      } 
      if (!iter.FinishEvent()) return;
   }
}
\end{verbatim}
\end{small}

Method \func{NewEvent()} will create and event header of the event id
\func{evntid} at the current iterator position in the buffer. 
In a similar way, method \func{NewSubevent()}
will put a subevent header of the specified arguments there. Access to the
data pointer after the last event or subevent header is done by
\func{iter.rawdata()} method. Finally, \func{FinishSubevent()} and
\func{FinishEvent()} will complete the subevent, or event definition, resp.

\section{File I/O}
MBS uses the {\tt .lmd}  ("List Mode Data") file format. 
Class \class{mbs::LmdFile} provides a
C++ interface for reading and writing such files.

To use \class{mbs::LmdFile} as input, or output transport of the module, 
classes \class{mbs::LmdInput} 
and \class{mbs::LmdOutput} were developed, resp. 

In general, to provide user-specific input/output capability over a port, 
one should implement the complete \class{dabc::Transport} interface, which includes 
event handling, queue organization, and a complex initialization sequence. 
All this is necessary for cases like socket or {\em InfiniBand} transports, 
but too complicated for simple cases as file I/O. Therefore, a special kind of
transport \class{dabc::DataIOTransport} was developed, 
which handles most of such complex tasks
and reduces the implementation effort to the 
relatively simple \class{dabc::DataInput} and
\class{dabc::DataOutput} interfaces.

Class \class{mbs::LmdOutput} inherits \class{dabc::DataOutput} 
and implements to save 
MBS events, if contained in \class{dabc::Buffer} objects, into an {\tt \*.lmd} file. 
In addition to \class{mbs::LmdFile} functionality, it allows to create
multiple files when a file size limit is exceeded. The class has following parameters:
\bbul
\item \param{MbsFileName}      - name of lmd file (including .lmd extension)  
\item \param{MbsFileSizeLimit} - size limit (in Mb) of single file, 0 - no limit
\ebul

Class \class{mbs::LmdInput} inherits \class{dabc::DataInput} and allows to read 
MBS events from {\tt \*.lmd} file(s) and to provide them over input ports into a module. 
It has following parameters:
\bbul
\item \param{MbsFileName} - name of lmd file (multicast symbols * and ? supported)  
\item \param{BufferSize}  - buffer size to read data
\ebul

\func{CreateDataInput()} and \func{CreateDataOutput()} methods were implemented in \class{mbs::Factory} class such, 
that a user can instantiate these classes 
via the \dabc~ plugin mechanism. 
  

Here is an example how the output file for a generator module can be configured:
   
\begin{small}
\begin{verbatim}
   ...
   dabc::mgr()->CreateModule("mbs::GeneratorModule", "Generator");
   dabc::Command* cmd = 
      new dabc::CmdCreateTransport("Generator/Output", mbs::typeLmdOutput);
   cmd->SetStr(mbs::xmlFileName, "output.lmd");
   cmd->SetInt(mbs::xmlSizeLimit, 100);
   dabc::mgr()->Execute(cmd);
   ...
\end{verbatim}
\end{small}
       
At first the module is created; then the type of
output transport and its parameters are set via command.        
       
Another example shows how several input files can be configured 
for a combiner module:
   
\begin{small}
\begin{verbatim}
   ...
   dabc::Command* cmd = 
      new dabc::CmdCreateModule("mbs::CombinerModule", "Combiner");
   cmd->SetInt(dabc::xmlNumInputs, 3);
   dabc::mgr()->Execute(cmd);
   
   for (unisgned n=0;n++;n<3) {
      cmd = new dabc::CmdCreateTransport(
         FORMAT(("Combiner/Input%u",n)), mbs::typeLmdInput);
      cmd->SetStr(mbs::xmlFileName, FORMAT(("input%u_*.lmd",n)));
      dabc::mgr()->Execute(cmd);
   }   
   ...
\end{verbatim}
\end{small}

In this example one creates  a module with 3 inputs and connects each input port 
with an {\tt *.lmd} file transport.


\section{Socket classes}

All communication with the MBS nodes is performed via tcp sockets. 
The \dabc~  base package {\tt libDabcBase.so} implements a number
of classes for general socket handling. The
main idea of these classes is to handle socket operations (creation, connection, sending, receiving, and error handling) by means of event processing callbacks. 

Class \class{dabc::SocketThread} organises the event loop
that handles such event signals produced by a socket.
Each system socket is assigned to an instance of \class{dabc::SocketProcessor}. 
The actual socket event processing is then done in virtual methods
of class \class{dabc::SocketProcessor} which has several subclasses for different 
kinds of sockets: 
\bbul
\item  - \class{dabc::SocketServerProcessor} - server socket for connection  
\item  - \class{dabc::SocketClientProcessor} - client socket for connection  
\item  - \class{dabc::SocketIOProcessor} - send/recv handling  
\ebul

One can use a \class{dabc::SocketThread} together with other kind of processors
like module classes, but not vice-versa: one cannot use socket processors inside other
thread types. Therefore, it is possible to run module with all 
socket transports in one single thread, if the socket thread for such module is created in
advance (see MBS generator example in section \ref{exa_mbs_eventsgenerator}).
  

\section{Server transport}

Class \class{mbs::ServerTransport} provides 
the functionalities of an {\em MBS transport server} or and  
{\em MBS stream server} in \dabc~. 
This is also a good example of the \class{dabc::SocketProcessor} classes.

\class{mbs::ServerTransport} is based on the generic class \class{dabc::Transport} and 
uses internally  two kinds of sockets: one socket for the connection handling,
and another "I/O" socket for sending data.
  
The server transport has following parameters:

\begin{tabular}{llll}
\hline
Name &  Type &  Dflt & Description  \\
\hline
\param{MbsServerKind} & str & Transport & kind of mbs server: "Transport" or "Stream" \\  
\param{MbsServerPort} & int & 6000 & server port number for socket connection \\
\hline
\end{tabular}

These parameters can be set in the XML configuration file like this:
\begin{small}
\begin{verbatim}
...
  <Module name="Generator">
    <Port name="Output">
      <MbsServerKind value="Transport"/>
      <MbsServerPort value="16020"/>
    </Port>
  </Module>
...
\end{verbatim}
\end{small}

To create such transport and connect it to the module's output port, 
the following code should be executed:
\begin{small}
\begin{verbatim}
  ...
  dabc::mgr()->CreateTransport("Generator/Output", 
                  mbs::typeServerTransport, "GeneratorThrd");
  ...
\end{verbatim}
\end{small}


Another possibility to specify these parameters consists in the 
command \comm{dabc::CmdCreateTransport} which may wrap such values:

\begin{small}
\begin{verbatim}
  ...
  dabc::Command* cmd = new dabc::CmdCreateTransport("Generator/Output", 
                         mbs::typeServerTransport, "MbsTransThrd");
  cmd->SetStr(mbs::xmlServerKind, mbs::ServerKindToStr(mbs::StreamServer));
  cmd->SetInt(mbs::xmlServerPort, mbs::DefaultServerPort(mbs::StreamServer) + 5);
  dabc::mgr()->Execute(cmd);
  ...
\end{verbatim}
\end{small}



\section{Client transport}

Class \class{mbs::ClientTransport} allows to connect \dabc~ with MBS. 
At the moment the MBS {\em transport} and {\em stream} servers are supported.  

Client transport has following parameters:

\begin{tabular}{llll}
\hline
Name &  Type &  Dflt & Description  \\
\hline
\param{MbsServerKind} & str & Transport & kind of mbs server: "Transport" or "Stream" \\  
\param{MbsServerName} & str & localhost & host name where mbs server runs  \\
\param{MbsServerPort} & int & 6000 & server port number for socket connection \\
\hline
\end{tabular}



To create client connection, the following piece of code should be used:

\begin{small}
\begin{verbatim}
  ...
  dabc::Command* cmd = new dabc::CmdCreateTransport("Combiner/Input0", 
                         mbs::typeClientTransport, "MbsTransThrd");
  cmd->SetStr(mbs::xmlServerKind, mbs::ServerKindToStr(mbs::StreamServer));
  cmd->SetStr(mbs::xmlServerName, "lxi010.gsi.de");
  cmd->SetInt(mbs::xmlServerPort, mbs::DefaultServerPort(mbs::StreamServer) + 5);
  dabc::mgr()->Execute(cmd);
  ...
\end{verbatim}
\end{small}

 

\section{Event generator}
\label{exa_mbs_eventsgenerator}

Class \class{mbs::GeneratorModule} is an example of a simple module  
which just fills buffers with random MBS 
events, and provides them to the output port. 
Schematically the implementation of this module looks like this:

\begin{small}
\begin{verbatim}
#include "dabc/ModuleAsync.h"

class GeneratorModule : public dabc::ModuleAsync {
protected:
   dabc::PoolHandle*       fPool;
   dabc::BufferSize_t      fBufferSize;
public:
   GeneratorModule(const char* name, dabc::Command* cmd = 0) :
      dabc::ModuleAsync(name, cmd)
   {
      ...
      fBufferSize = GetCfgInt(dabc::xmlBufferSize, 16384, cmd);
      fPool = CreatePoolHandle("Pool", fBufferSize, 10);
      CreateOutput("Output", fPool, 5);
   }

   virtual void ProcessOutputEvent(dabc::Port* port)
   {
      dabc::Buffer* buf = fPool->TakeBuffer(fBufferSize);
      FillRandomBuffer(buf);
      port->Send(buf);
   }
};
\end{verbatim}
\end{small}

In the module constructor a pool handle is created, declaring a required
memory pool with 10 buffers of defined size \func{fBufferSize}. 
The buffer size is taken from a configuration parameter with name 
\func{dabc::xmlBufferSize} (this string constant is 
predefined as "BufferSize").
When the output port is created, this pool handle and
a default queue size is specified.

The only virtual method implemented for generator module is
\func{ProcessOutputEvent()}. This function is called every time when a free 
buffer slot appears in the port output queue. Thus, when the module starts
processing, this call will be immediately executed 
$N$ times (size of output queue, here 5), because there are $N$ empty 
entries in the queue.
The only action here is to take a new buffer from the memory pool, 
fill it with random events and send it to output port.    

The actual \class{mbs::GeneratorModule} is part of the {\tt libDabcMbs.so} library and has
following parameters:

\begin{tabular}{llll}
\hline
Name &  Type &  Dflt & Description  \\
\hline
\param{NumSubevents} & int & 2 &  number of subevents in generated event  \\   
\param{FirstProcId}  & int & 0 & value of procid field of first subevent \\
\param{SubeventSize} & int & 32  &  size of rawdata in subevent            \\
\param{Go4Random}    & bool & true & is raw data filled with random value  \\
\param{BufferSize}   & int & 16384 & server port number for socket connection \\
\hline
\end{tabular}
  
There following function \func{StartMbsGenerator()} instantiates and runs 
the generator module.
It also demonstrates how a thread of type \class{dabc::SocketThread} can be created
and used by both a module and a transport object. 
\begin{small}
\begin{verbatim}
extern "C" void StartMbsGenerator()
{
  dabc::mgr()->CreateThread("GenerThrd", dabc::typeSocketThread);
  dabc::mgr()->CreateModule("mbs::GeneratorModule", "Generator", "GenerThrd");
  dabc::mgr()->CreateTransport("Generator/Output", mbs::typeServerTransport, "GenerThrd");
  dabc::mgr()->StartModule("Generator");
}
\end{verbatim}
\end{small}

To run the generator module with all default parameters, 
this simple XML file is sufficient:   
\begin{small}
\begin{verbatim}
<?xml version="1.0"?>
<dabc version="1">
  <Context host="lxi009" name="Server">
    <Run>
      <lib value="libDabcMbs.so"/>
      <func value="StartMbsGenerator"/>
    </Run>
  </Context>
</dabc>
\end{verbatim}
\end{small}

Besides one may specify all module and transport parameters explicitely here:
\begin{small}
\begin{verbatim}
<?xml version="1.0"?>
<dabc version="1">
  <Context host="lxi009" name="Server">
    <Run>
      <lib value="libDabcMbs.so"/>
      <func value="StartMbsGenerator"/>
    </Run>
    <Module name="Generator">
       <NumSubevents value="3"/>
       <FirstProcId value="77"/>
       <SubeventSize value="128"/>
       <Go4Random value="false"/>
       <BufferSize value="16384"/>
       <Port name="Output">
          <OutputQueueSize value="5"/>
          <MbsServerKind value="Stream"/>
          <MbsServerPort value="6006"/>
        </Port>
     </Module>
  </Context>
</dabc>
\end{verbatim}
\end{small}

Example file {\tt \$DABCSYS/application/mbs/GeneratorTest.xml}  
demonstrates the usage of a generator module. 


\section{MBS event building}
Class \class{mbs::CombinerModule} allows to combine events from
several running MBS systems. It has following parameters:

\begin{tabular}{llll}
\hline
Name &  Type &  Dflt & Description  \\
\hline
\param{BufferSize}   & int  & 16384  &  buffer size of output data \\
\param{NumInputs}    & int  & 2      &  number of mbs data sources  \\   
\param{DoFile}       & bool & false  &  create LMD file store for combined events \\
\param{DoServer}     & bool & false  &  create MBS server to provide data further   \\
\hline
\end{tabular}

The module implements two optional output ports: 
for file storage (port name "FileOutput"), and for 
providing data further over an MBS server (port name "ServerOutput"). 

Function \func{StartMbsCombiner()} initializes the combiner module and starts
data taking. The following example configuration file {\tt \$DABCSYS/applications/mbs/Combiner.xml} shows how to 
configure a combiner module reading from three MBS transport servers: 

\begin{small}
\begin{verbatim}
<?xml version="1.0"?>
<dabc version="1">
  <Context host="localhost" name="Combiner">
    <Run>
      <lib value="libDabcMbs.so"/>
      <func value="StartMbsCombiner"/>
      <logfile value="combiner.log"/>
    </Run>
    <Module name="Combiner">
       <NumInputs value="3"/>
       <DoFile value="false"/>
       <DoServer value="true"/>
       <BufferSize value="16384"/>
       <Port name="Input0">
          <InputQueueSize value="5"/>
          <MbsServerKind value="Transport"/>
          <MbsServerName value="lxi009"/>
          <MbsServerPort value="6000"/>
       </Port>
       <Port name="Input1">
          <InputQueueSize value="5"/>
          <MbsServerKind value="Transport"/>
          <MbsServerName value="lxi010"/>
          <MbsServerPort value="6000"/>
       </Port>
       <Port name="Input2">
          <InputQueueSize value="5"/>
          <MbsServerKind value="Transport"/>
          <MbsServerName value="lxi011"/>
          <MbsServerPort value="6000"/>
       </Port>
       <Port name="FileOutput">
          <OutputQueueSize value="5"/>
          <MbsFileName value="combiner.lmd"/>
          <MbsFileSizeLimit value="128"/>
        </Port>
       <Port name="ServerOutput">
          <OutputQueueSize value="5"/>
          <MbsServerKind value="Stream"/>
        </Port>
     </Module>
  </Context>
</dabc>
\end{verbatim}
\end{small}

\section{\mbs\ upgrade for \dabc}
This section is rather for the \mbs\ programmer than for application programmers.
To have minimal changes, we use standard collector and transport. Two changes:
\subsection{Increased buffer size support}
This is done in a completely compatible way. 
For the following see data structures \paref{prog:mbs-file-header} and \paref{prog:mbs-buffer-header}.
The only problem is the 16 bit \decl{i\_used} field in the old buffer header \decl{s\_bufhe} structure (new \decl{iUsed}) keeping the number of 16 bit data words (behind buffer header). 
The other 16 bits are used for event spanning.
With a new rule we store this number in 32 bit field \decl{l\_free[2]}
, now \decl{iUsedWords}. Only if old \decl{l\_dlen}, now \decl{iMaxWords} is less equal {\tt MAX\_\_DLEN} defined as {\tt (32K-sizeof(bufhe))/2} this number is also stored in {\tt i\_used} 
(\decl{iUsed}) as before. Modifications have to be made in all \mbs\ modules accessing {\tt i\_used}. Modules outside \mbs\ can be modified on demand to support large buffers. Current buffers still can be handled without change.\\
When \mbs\ writes large buffer files only the used part of the file header is written. Number of 16 bit words behind buffer header structure is stored in {\tt filhe\_used}.
Event API \decl{f\_evt} is updated to handle large buffers on input.
Note: By setup the number of buffers per stream can be set to one. This suppresses event spanning.
Large buffers can be used by standard \mbs.
\subsection{Variable sized buffers}
As a second step variable sized buffers are implemented. They get a new type 100. The allocated buffers are still fixed length as before. However, the \mbs\ transport would write only the used part of the buffers to clients. Processing these buffers a module must first read the header, then get the used size from {\tt iUsedWords} (old {\tt l\_free[2]}) and read the rest. Modules outside \mbs\ must be modified to process such buffers.
In \mbs, after stream buffers are created, buffer types are set to 100 by a new command \comm{enable dabc} in transport. This command also sets the transport synchronous mode. In this mode transport processes streams only if a client is connected.
\subsection{New LMD file format}
With \dabc\ as event builder for \mbs\ there is no need to write files in \mbs. This gives more freedom to design a new file format. This format will be written by \dabc\ and read by {\tt fLmd} functions (get event). The format is quite simpler than the old one, because it has no buffer structure causing so much complications by event spanning. The data elements itself, mainly the events, remain unchanged.

A file has a file header as before, but with a fixed size part and a variable part 
(size {\tt iUsedWords}).

Behind the header follow data elements with {\tt sMbsHeader} headers (length, type, subtype) allowing to identify and process or skip them. Elements must be sized in 4 byte units.
Besides event data, time stamps may be inserted from the original \mbs\ formatted buffers to preserve this time information. Writing/reading such a file is very straight forward. The file header contains the number of data elements (\verba{iElements}) and the maximum size of elements (\verba{iMaxWords}). This information is collected throughout the file writing and written on close into the file header. The file header is an {\tt sMbsFileHeader} structure.

The file size is no longer restricted to 2GB. Optionally an element index is written at the end of the file. This allows for random access of elements in the file through this index table. The table itself has 32-bit values for the element offsets (in 32-bit). It can therefore address offsets up to 16GB in the file. If larger files are needed, the table can be created with 64-bit values giving unlimited addressing.

Note: This file format needs the \verba{rewind} file function because the file header must be rewritten to store {\tt iMaxWords}, {\tt iElements}, and optionally the offset of the index table. This function is currently not implemented in the {\tt RFIO} package, but will be done.

\lsubsection{prog:mbs-data-struct}{\mbs\ data structures}
All structures are defined independent on endianess. When bytes must be swapped, always 4 bytes are swapped. Fields 8 bytes long must be handled separately.
Smaller items must be accessed by mask\&shift.
This makes code independent of endian.
\subsubsection{Connect to \mbs\ transport}
Structure used to talk between client and transport server.
Client connects to server (\mbs) and reads this structure first.
Structure maps the \decl{sMbsTransportInfo} info buffer.
{\small \begin{verbatim}
typedef struct{
  uint32_t iEndian;      // byte order. Set to 1 by sender 
  uint32_t iMaxBytes;    // maximum buffer size                   
  uint32_t iBuffers;     // buffers per stream (should be 1 for DABC mode)
  uint32_t iStreams;     // number of streams (=0 for DABC mode)                   
} sMbsTransportInfo;
\end{verbatim}
}

\lsubsubsection{prog:mbs-buffer-header}{Buffer header}
Buffer header, maps \decl{s\_bufhe}, some fields used in different way.
The main difference is the usage of \decl{iUsedWords} for the data length.
{\small \begin{verbatim}
typedef struct{
  uint32_t iMaxWords;    // compatible with s_bufhe (total buffer size - header)
  uint32_t iType;        // compatible with s_bufhe, low=type (=100), high=subtype
  uint32_t iUsed;        // not used for iMaxWords>MAX__DLEN (16360), low 16bits only
  uint32_t iBuffer;      // compatible with s_bufhe
  uint32_t iElements;    // compatible with s_bufhe
  uint32_t iTemp;        // Used volatile
  uint32_t iTimeSpecSec; // compatible with s_bufhe (2*32bit) (struct timespec)
  uint32_t iTimeSpecNanoSec; // compatible with s_bufhe (2*32bit) (struct timespec)
  uint32_t iEndian;      // compatible with s_bufhe free[0]
  uint32_t iWrittenEndian; // LMD__ENDIAN_BIG, LMD__ENDIAN_LITTLE, LMD__ENDIAN_UNKNOWN
  uint32_t iUsedWords;   // total words without header, free[2]
  uint32_t iFree3;       // free[3]
} sMbsBufferHeader;
\end{verbatim}
}
\lsubsubsection{prog:mbs-file-header}{File header}
File header, maps \decl{s\_bufhe}, some fields used in different way.
{\small \begin{verbatim}
typedef struct{
  uint32_t iMaxWords;    // Size of largest element in file
  uint32_t iType;        // compatible with s_bufhe, low=type (=100), high=subtype
  lmdoff_t iTableOffset; // optional offset to element index table in file
  uint32_t iElements;    // Number of elements in file
  uint32_t iOffsetSize;  // Offset size, 4 or 8 [bytes]
  uint32_t iTimeSpecSec; // compatible with s_bufhe (2*32bit) (struct timespec)
  uint32_t iTimeSpecNanoSec; // compatible with s_bufhe (2*32bit) (struct timespec)
  uint32_t iEndian;      // compatible with s_bufhe free[0]
  uint32_t iWrittenEndian; // LMD__ENDIAN_BIG, LMD__ENDIAN_LITTLE, LMD__ENDIAN_UNKNOWN
  uint32_t iUsedWords;   // total words following header, free[2]
  uint32_t iFree3;       // free[3]
} sMbsFileHeader;
\end{verbatim}
}
\subsubsection{Data element structures}
\bbul
\item Time stamp
{\small \begin{verbatim}
typedef struct{
  uint32_t iMaxWords; 
  uint32_t iType; 
  uint32_t iTimeSpecSec;
  uint32_t iTimeSpecNanoSec;
} sMbsTimeStamp;
\end{verbatim}
}
\item Common data item header
{\small \begin{verbatim}
typedef struct{
  uint32_t iWords;       // following data words
  uint32_t iType;        // compatible with s_ve10_1, low=type (=10), high=subtype
} sMbsHeader;
\end{verbatim}
}
\item \mbs\ event header (type 10,1)
{\small \begin{verbatim}
typedef struct{
  uint32_t iWords;       // data words + 4
  uint32_t iType;        // compatible with s_ve10_1, low=type (=10), high=subtype
  uint32_t iTrigger;
  uint32_t iEventNumber;
} sMbsEventHeader;
\end{verbatim}
}
\item \mbs\ subevent header
{\small \begin{verbatim}
typedef struct{
  uint32_t iWords;       // data words + 2
  uint32_t iType;        // compatible with s_ves10_1, low=type (=10), high=subtype
  uint32_t iSubeventID;  // 2 low bytes=procid, next byte=subcrate, high byte control 
} sMbsSubeventHeader;
\end{verbatim}
}
\ebul
\subsubsection{Some fixed numbers}
{\small \begin{verbatim}
#define LMD__TYPE_FILE_HEADER_101_1   0x00010065
#define LMD__TYPE_EVENT_HEADER_10_1   0x0001000a
#define LMD__TYPE_FILE_INDEX_101_2    0x00020065
#define LMD__TYPE_BUFFER_HEADER_10_1  0x0001000a
#define LMD__TYPE_BUFFER_HEADER_100_1 0x00010064
#define LMD__TYPE_TIME_STAMP_11_1     0x0001000b
#define LMD__INDEX         1
#define LMD__OVERWRITE     1
#define LMD__LARGE_FILE    1
#define LMD__BUFFER        1
#define LMD__NO_INDEX      0
#define LMD__NO_OVERWRITE  0
#define LMD__NO_LARGE_FILE 0
#define LMD__NO_BUFFER     0
#define LMD__NO_VERBOSE    0
#define LMD__VERBOSE       1
#define LMD__ENDIAN_BIG    2
#define LMD__ENDIAN_LITTLE 1
#define LMD__ENDIAN_UNKNOWN 0
\end{verbatim}
}
\subsection{\mbs\ update for DIM control}
\subsubsection{New or modified files}
\strong{New files:}
\bdes
\item [f\_dim\_server.c, f\_dim\_server.h] : all DIM functions.
\item [dimstartup.sc, dimshutdown.sc] : for single node
\item [prmstartup.sc, prmshutdown.sc, dimremote\_exe.sc] : for multi node, propagate \keyw{DIM\_DNS\_NODE}
\item [m\_launch.c] : fork programs to appear without path in ps output.
\item [m\_cmd2xml.c] : Generate xml command description file from output of show command
\edes
\strong{Modified:}
\bdes
\item [alias.com] : add launch alias
\item [m\_prompt.c]
\item [m\_dispatch.c]
\item [m\_msg\_log.c]
\item [f\_ifa.c, f\_ifa.h]
\item [f\_mg\_msg\_output.c]
\item [f\_mg\_msg\_thread.c] : in /mbs/v51 has argument which is not specified in call!
\item [f\_pr\_reset.c] : kill processes in defined order.
\item [f\_stccomm.c] : Socket created in stc\_createserver will be shut down and closed by stc\_close.
\item [m\_wait\_for.c] : Dont wait for zombies.
\item [remote\_exe.sc] : use m\_launch to run program.
\item [Makefile] : modules using DIM must include DIM path, all programs must link DIM library.
\edes
\subsubsection{f\_stccomm}
On the server side \verba{stc\_createserver} fills structure \verba{s\_tcpcomm} with socket number\\ (\verba{s\_tcpcomm.sock\_rw}). \verba{stc\_acceptclient} returnes socket number of connection. This socket must be closed by \verba{stc\_discclient(socket)}. \verba{stc\_close} shuts down and closes socket \verba{s\_tcpcomm.socket}, not \verba{s\_tcpcomm.sock\_rw}. 

Therefore a server side \verba{stc\_close} did not close the server socket. This has been changed in that \verba{s\_tcpcomm.socket} is now equal \verba{s\_tcpcomm.sock\_rw}. When a server called \verba{stc\_close} no more accept is possible as opposed to the current behavior.

On the client side a \verba{stc\_connectserver} returns the socket as well as setting\\ \verba{s\_tcpcomm.socket}. Therefore in this case \verba{stc\_close} was always shutting down and closing the socket, whereas \verba{stc\_discclient(socket)} closes the socket.
\subsubsection{MBS launcher}
\begin{verbatim}
  launch <program> [<program path>] [. <program args>]
\end{verbatim}
If no program path is given, \keyw{MBSROOT} bin directory is assumed. Note that in the launched program environment PWD is the path from where \verba{m\_launch} was called, whereas the current path is the one of the program. Therefore program must \verba{chdir(getenv("PWD"))} to work on the expected directory.
\subsubsection{\mbs\ DIM commands and parameters}
The parameters and commands follow the \dabc\ naming conventions. The DIM command \comm{MbsCommand} is generic. The argument string is any \mbs\ command. All \verba{.scom} files are provided as DIM commands. New program \verba{cmd2xml} generates an xml file with the \dabc\ formatted description of all MBS commands.
\subsubsection{DIM control modes}
\mbs\ can be controlled through DIM in two modes: single and multimode. In single mode the dispatcher is a DIM command server, the message logger a DIM parameter server for messages and status information (selected from DAQ status). In multinode mode the prompter is the DIM command server and the message loggers are status servers. The master message logger also is the DIM message server.

The table \paref{prog:mbs-modes} shows an overview of the different operation modes.
\begin{table}[h]
\begin{center}
\begin{tabular}{|p{2.0cm}|p{4.0cm}|p{4.0cm}|p{2cm}|}      \hline
Mode   & Interactive      & DIM GUI     & Remote GUI \\ \hline
Single & Dispatcher:TTY   & DIM command & - \\ \hline
       & Logger: TTY,file & DIM status+message, file & - \\ \hline
Multi  & Dispatcher:TCP   & TCP & TCP \\ \hline
       & Prompter:TTY     & DIM command & TCP \\ \hline
       & MasterLogger:TTY+file & file & file \\ \hline
       & Msg server       & DIM status+message & Msg server \\ \hline
       & TCP inputs       & TCP inputs & TCP inputs \\ \hline
       & SlaveLogger:TCP  & TCP,DIM status & TCP \\ \hline
\end{tabular}
\end{center}
\caption{\mbs\ operation modes.}
\label{prog:mbs-modes}
\end{table}
\subsubsection{Single node mode}
There are two new scripts to start and shutdown a single node MBS: \\
\verba{dimstartup.sc} and \verba{dimshutdown.sc}. \\
These are called by \verba{rsh} from the GUI node. Arguments are the path of \keyw{MBSROOT} and the user working path. For starting the DIM server also the DIM name server node is passed.
\begin{verbatim}
dimstartup.sc $MBSROOT $PWD $DIM_DNS_NODE 
\end{verbatim}
launches \verba{m\_dispatch -dim} after waiting for all 60xx sockets closed.
\begin{verbatim}
dimshutdown.sc $MBSROOT $PWD 
\end{verbatim}
calls \verba{m\_remote reset -l}.
When dispatcher is started with \keyw{-dim} message logger is started with argv[1] = \keyw{dim} (otherwise \keyw{task}). Then the DIM commands are defined and dispatcher goes into \verba{pause()}
 loop (needed with non threaded DIM version for keep alive signals).

When message logger is started with argv[1] = \keyw{dim} it creates the DIM parameters and starts a thread to update these every second. One DIM parameter is used for the messages and updated when a message arrives. Messages from local tasks are received in a thread (\verba{f\_mg\_msg\_thread}) and either sent to master message logger (when this one is slave) or processed by \verba{f\_mg\_msg\_output}. This function updates the DIM message parameter when in DIM mode, sends message to connected remote message client or prints it if not, and writes log file.
\subsubsection{Multi node mode}
In multi node mode the MBS nodes are controlled through one master node where a prompter is running. The prompter can be started (and stopped) interactively, or from a remote node by script. There are two new scripts to start and shutdown a multi node MBS from a remote node (GUI): \\
\verba{prmstartup.sc} and \verba{prmshutdown.sc}. \\
These are called by \verba{rsh} from the GUI node. Arguments are the path of \keyw{MBSROOT} and the user working path. For starting the DIM server also the DIM name server node is passed.
\begin{verbatim}
prmstartup.sc $MBSROOT $PWD $DIM_DNS_NODE $REMOTE_NODE 
\end{verbatim}
launches \verba{m\_prompt -dim -r <remotenode>} after waiting for all 60xx sockets closed.
\begin{verbatim}
prmshutdown.sc $MBSROOT $PWD 
\end{verbatim}
calls \verba{m\_remote reset -l task=m\_prompt}. With \verba{m\_wait\_for -task m\_prompt} it waits for prompter to be stopped, then calls \verba{m\_remote reset} (all nodes from \verba{node\_list.txt}).
For the message logger modes see table \paref{prog:mbs-msg-modes}.
 When prompter is started with \keyw{-dim} it starts the master message logger with argv[1] = \keyw{masterdim }(otherwise = \keyw{master}). When prompter is started with the -r <remote argument>, the remote node name is passed as argv[2] to the message logger.

All dispatchers are started in the MBS prompter by \verba{f\_ifa\_init(NodeList,DimNameServer)} function. It calls per node from \verba{NodeList f\_ifa\_remote} which starts the dispatcher in function \verba{f\_ifa\_rsh\_proc\_start} by script \verba{dimremote\_exe.sc} (DIM mode) or \verba{remote\_exe.sc} (normal mode).
When prompter is started with \keyw{-dim} then the DIM command server is started. Otherwise it starts TCP server on port 6006 (if started with the -r <remotenode> option) waiting for connection of GUI client and reading commands, or reading commands from terminal.
\begin{verbatim}
dimremote_exe.sc $MBSROOT $PWD $DIM_DNS_NODE $PROMPTER_NODE 
\end{verbatim}
launches \verba{m\_dispatch -dim -<prompternode>}. 
The dispatchers start their message logger with the same argument (prompter node) and 
optional argv[2] = \keyw{slavedim}. On the prompter node, however, the message logger should already run (started by prompter).

Because started with argv[2] = -<prompternode> the dispatchers then start a TCP server waiting on port 6004 for connection of prompter. When prompter connects, dispatcher sends process id. Then waits for commands. Command completion is sent back. Prompter may terminate, start again and connect.

The master logger starts a server in a thread waiting on port 6005 for connections of message logger slaves. For each slave a new thread is started waiting for messages of the slave. These threads are protected by mutex. Only one thread can write into logfile. However, slaves do not write logfile. If prompter was started from remote, master message logger starts server on port 6007 waiting for remote message client to connect.
\begin{table}[h]
\begin{center}
\begin{tabular}{|p{3.3cm}|p{0.6cm}|p{0.7cm}|p{3.6cm}|p{5.9cm}|}      \hline
argv & TCP & Slave & DIM & remote \\ \hline
none & 0 & 0 & - & - \\ \hline
\keyw{task} & 0 & 0 & - & - \\ \hline
\keyw{master} & 1 & 0 & - & pth\_server$\rightarrow$pth\_links \\ \hline
\keyw{dim} & 0 & 0 & msg,status,pth\_dim\_serv &   \\ \hline
\keyw{masterdim} & 1 & 0 & msg,status,pth\_dim\_serv & pth\_server$\rightarrow$pth\_links \\ \hline
masternode & 1 & 1 & - & connect masternode \\ \hline
\keyw{masterdim} any & 1 & 0 & msg,status,pth\_dim\_serv & pth\_server$\rightarrow$pth\_links \\ \hline
masternode \keyw{slavedim} & 1 & 1 & status,pth\_dim\_serv & connect masternode \\ \hline
\keyw{master} remotenode & 1 & 0 & msg,status,pth\_dim\_serv & pth\_rem\_serv,pth\_server $\rightarrow$pth\_links \\ \hline
\end{tabular}
\end{center}
\caption{\mbs\ Message logger modes.}
\label{prog:mbs-msg-modes}
\end{table}
If remote node is specified, \verba{pth\_rem\_serv} thread waits for connection of a message client. After connection global \verba{l\_tcp\_chan\_rem} is set and \verba{f\_mg\_msg\_output} send messages to that socket.
As TCP master the \verba{pth\_server} thread waits for connections of message slaves. After connection starts \verba{pth\_links} thread to read messages and process in \verba{f\_mg\_msg\_output} (mutex locked).
As TCP slave connect to <masternode>, set global \verba{l\_tcp\_chan}. In DIM mode create services and start \verba{pth\_dim\_serv} to update every second. In all cases \verba{f\_mg\_msg\_thread} is called where in slave mode messages are sent to the master, otherwise processed by \verba{f\_mg\_msg\_output}.

\subsubsection{MBS controlled by DIM}
Graphics on Eigene Dateien/experiments/mbs
\figpdf{prog-mbs-dim1}{Single node \mbs\ controlled by DIM.}{htb}{0}{1}
\figpdf{prog-mbs-tcp}{Multi node \mbs\ controlled by TCP.}{htb}{0}{1}
\figpdf{prog-mbs-dim2}{Multi node \mbs\ controlled by DIM.}{htb}{0}{1}
\section{List of icons}
\icon{browser} browser Browser.\\
\icon{comicon} comicon Command.\\
\icon{conndsp} conndsp Connect single MBS\\
\icon{connprm} connprm Connect MBS prompter\\
\icon{control} control Test, shell script\\
\icon{controlmbs} controlmbs \dabc\ shell script\\
\icon{controldabc} controldabc \mbs\ shell script\\
\icon{dabcconfig} dabcconfig Configure\\
\icon{dabcicon} dabcicon \dabc\ launcher\\
\icon{dabcmbsicon} dabcmbsicon \dabc\ \mbs\ launcher\\
\icon{dabcstart} dabcstart Start acquisition\\
\icon{dabcstop} dabcstop Hold acquisition\\
\icon{disconn} disconn Shut down all \\
\icon{exitall} exitall Exit all\\
\icon{fileclose} fileclose Close\\
\icon{filesave} filesave Save\\
\icon{histowin} histowin Histogram panel\\
\icon{info} info Show acquisition\\
\icon{infowin} infowin Info panel\\
\icon{logwin} logwin Log panel\\
\icon{mbsconfig} mbsconfig Configure\\
\icon{mbsicon} mbsicon \mbs\ launcher\\
\icon{mbsstart} mbsstart Start acquisition\\
\icon{mbsstop} mbsstop Stop acquisition\\
\icon{meterwin} meterwin Rate meter panel\\
\icon{paramwin} paramwin Parameter table\\
\icon{rshmbs} rshmbx \mbs\ remote shell script\\
\icon{savewin} savewin Save settings\\
\icon{statewin} statewin State panel\\
\icon{usergraphics} usergraphics Graphics panel\\
\icon{usericon} usericon Parameter selection panel\\
\icon{usericonblue} usericonblue User panel\\
\icon{usericonred} usericonred User panel\\
\icon{usericongreen} usericongreen User panel\\
\icon{usericonyellow} usericonyellow User panel\\
\icon{user} user Windows\\
\icon{windowclose} windowclose Windows close\\
\icon{windowblue} windowblue Graphics window\\
\icon{windowred} windowred Graphics window\\
\icon{windowgreen} windowgreen Graphics window\\



[programmer/prog-exa-roc.tex]

\section{Overview}

CBM readout controller (ROC) is FPGA-based board, which is aimed to readout
nXYTER chip and transport data over Ethernet to PC. Software package ROClib 
provides basic functionality to readout data from ROC.
 
To support usage of ROC in DABC, following classes were designed:
\bbul
\item \class{roc::Device}  device class, wrapper for SysCoreController 
\item \class{roc::Transport}  access to SysCoreBoard functionality 
\item \class{roc::CombinerModule}  module to combine data from several ROCs in single output  
\item \class{roc::CalibrationModule}  module to calibrate time scale in ROC data  
\item \class{roc::ReadoutApplication}  application to perform readout from ROC boards   
\item \class{roc::Factory}  factory class to organize plugin 
\ebul


\section{Device and transport}

ROC device class \class{roc::Device} inherits from two classes: \class{dabc::Device} and \class{SysCoreControl},
where \class{SysCoreControl} provides simultaneous access to several ROC boards. 
Normally instance of device class corresponds to physical device or board, 
but here device object used rather as central collection of \class{SysCoreBoard} objects and
as thread provider.

Each instance of \class{roc::Transport} has a pointer to \class{SysCoreBoard} object,
over which data taking from specific ROC is performed. 
Implementation of class \class{roc::Transport} based on \class{dabc::DataTransport} class.
Class \class{dabc::DataTransport} uses event loop mechanism and does not requires explicit thread.
This feature allows to run several instances of such transports in the same thread.
In ROC case all \class{roc::Transport} instances uses thread of \class{roc::Device}.     
Lets try to describe how class \class{roc::Transport} is working.

In the beginning of event loop (when module starts) \func{roc::Transport::StartTransport} is called, which
is used to call \func{SysCoreBoard::startDaq} to start data taking. After that
event loop consist from subsequent calls of \func{Read\_Size}, \func{Read\_Start} and \func{Read\_Complete}
functions, derived from \class{dabc::DataTransport} class. 

Aim of \func{Read\_Size} method is to define size of next buffer, required for data reading. 
In case of \class{roc::Transport} this size
is fixed and defined by configuration parameter:  

\begin{verbatim}
unsigned roc::Transport::Read_Size()
{
   return fBufferSize;
}
\end{verbatim}
    
When system deliver buffer of requested size, \func{Read\_Start} function is called to start reading 
of that buffer from data source. \class{SysCoreBoard} internally has its own buffer, therefore call 
\func{SysCoreBoard::requestData} either inform that required number of messages already received or
one should wait. Waiting in thread logic will mean that thread should be blocked and 
cannot be used by other transport. Therefore, another approach is used - ROClib will provide 
call back of virtual \func{SysCoreControl::DataCallBack} method when required amount of data is there.
Implementation of method is looks like:

\begin{verbatim}
unsigned roc::Transport::Read_Start(dabc::Buffer* buf)
{
   int req = fxBoard->requestData(fReqNumMsgs);
   if (req==2) return dabc::DataInput::di_CallBack;
   if (req==1) return dabc::DataInput::di_Ok;
   return dabc::DataInput::di_Error;
}
\end{verbatim}

In case when data already exists in internal buffer of \class{SysCoreBoard} object, \keyw{dabc::DataInput::di\_Ok} 
is returned and than immediately \func{Read\_Complete} will be called, which finally fill output buffer:

\begin{verbatim}
unsigned roc::Transport::Read_Complete(dabc::Buffer* buf)
{
   unsigned fullsz = buf->GetDataSize();
   if (!fxBoard->fillData((char*) buf->GetDataLocation(), fullsz)) 
      return dabc::DataInput::di_SkipBuffer;
   if (fullsz==0) 
      return dabc::DataInput::di_SkipBuffer;
   buf->SetTypeId(roc::rbt_RawRocData);
   buf->SetDataSize(fullsz);
   return dabc::DataInput::di_Ok;
}
\end{verbatim}

Returned from function \func{Read\_Start} value \keyw{dabc::DataInput::di\_CallBack} indicates 
that event loop should be suspended. When required amount of data received by ROClib,
it produces call of \func{SysCoreControl::DataCallBack} method, which is reimplemented in \class{roc::Device} 
class and calls following method of \class{roc::Transport}: 

\begin{verbatim}
void roc::Transport::ComplteteBufferReading()
{
   unsigned res = Read_Complete(fCurrentBuf);
   Read_CallBack(res);
}
\end{verbatim}

While required amount of data is received,
one only retrieves it with the same \func{Read\_Complete} method and 
reactivate event loop by calling \func{Read\_CallBack}.   


\section{Combiner module}

Class \class{roc::CombinerModule} is designed to combine data from several ROC boards in one MBS event. It also performs sorting of
data according timestamp, resolves last epoch bits and fixes several coding errors (class \class{SysCoreSorter} is
used for this).

Module has following configuration parameters:
\bbul
\item \param{NumRocs}     - number of ROC boards, connected to combiner [default 1]  
\item \param{BufferSize}  - size of buffer, used to read data from ROCs [default 16386]
\item \param{NumOutputs}  - number of outputs [default 2]
\ebul

As output MBS events are provided. Each MBS event contain ROC messages between two sync markers.
For each ROC separate MBS event allocated. Field iSubcrate contains ROC id.


\section{Calibration module}

Class \class{roc::CalibrationModule} perform calibration of time scale for all
ROCs and merging all messages in single data stream. As output, MBS event with single
subevent is produced. 

Module has following configuration parameters:
\bbul
\item \param{NumRocs}     - number of ROC boards, which should be provided in MBS event [default 2]  
\item \param{BufferSize}  - size of buffer, used to produce output data [default 16386]
\item \param{NumOutputs}  - number of outputs [default 2]
\ebul
 

\section{Readout application}

The main aim of \class{roc::ReadoutApplication} class is configure and run application,
which combines readout of data from several ROCs, store it into lmd file and optionally 
create mbs stream server to observe data from remote Go4 GUI.
It has following configuration parameters:

\bbul
\item \param{NumRocs}   - number of ROC boards  
\item \param{RocIp0}, \param{RocIp1}, \param{RocIp2}, ... - addresses (IP or nodname) of ROC boards
\item \param{DoCalibr}  - defines calibration mode (see further)
\item \param{BufferSize} - size of buffer
\item \param{NumBuffers} - number of buffers
\item \param{MbsServerKind} - kind of MBS server (None, Stream, Transport)
\item \param{RawFile} - name of lmd file to store combined data
\item \param{CalibrFile} - name of lmd file to store calibrated data
\item \param{MbsFileSizeLimit} - maximum size of each file, in Mb
\ebul

Three calibration modes are supported:
\bbul
\item \param{DoCalibr}=0 - Only CombinerModule is instantiated, which produces kind of ROC "raw" data 
\item \param{DoCalibr}=1 - Both CombinerModule and CalibrationModule are instantiated   
\item \param{DoCalibr}=2 - Only CalibrationModule, used to convert raw data from lmd files into calibrated format.
\ebul

In all modes output in form of raw or (and) calibrated data can be stored in lmd file(s),
defined by \param{RawFile} and \param{CalibdFile} parameters respectively. Last mode is special case,
when \param{RawFile} specifies not output but input file for the calibration module.


\section{Factory}

Factory class \class{roc::Factory} implements several methods to create
ROC-specific application, device and modules.    


\section{Source and compilation}

Source code for all classes can be found in \$DABCSYS/plugins/roc directory. 
Compiled library libDabcKnut.so should be found in \$DABCSYS/lib directory.
If one need to modify some code in this library, one should copy sources in 
user directory and call "make" in this directory. In this case library can be found
in directory like \$ARCH/lib, where \$ARCH is current CPU architecture (for instance, i686).

 
\section{Running ROC application}

To run readout application, approprite xml configuration file should be created.
There are two examples of configuration files in \$DABCSYS/applications/roc.
In Readout.xml one founds example of readout from 3 ROCs: 

\begin{verbatim}
<?xml version="1.0"?>
<dabc version="1">
<Context name="Readout">
  <Run>
    <lib value="libDabcMbs.so"/>
    <lib value="libDabcKnut.so"/>
    <logfile value="Readout.log"/>
  </Run>
  <Application class="roc::Readout">
    <DoCalibr value="0"/>
    <NumRocs value="3"/>
    <RocIp0 value="cbmtest01"/>
    <RocIp1 value="cbmtest02"/>
    <RocIp2 value="cbmtest04"/>
    <BufferSize value="65536"/>
    <NumBuffers value="100"/>
    <TransportWindow value="30"/>
    <RawFile value="run090.lmd"/>
    <MbsServerKind value="Stream"/>
    <MbsFileSizeLimit value="110"/>
  </Application>
 </Context>
</dabc>
\end{verbatim}

While this is single-node application, to run it, dabc\_run executable can be used:
\verba{dabc\_run Readout.xml}. dabc\_run executable will load specified libraries,
create application, configure it and switch system in running mode.   

In Calibr.xml shown special case of configuration to convert raw data 
into calibrated files without running any real DAQ. 

\begin{verbatim}
<?xml version="1.0"?>
<dabc version="1">
<Context name="Calibr">
  <Run>
    <lib value="libDabcMbs.so"/>
    <lib value="libDabcKnut.so"/>
    <logfile value="Calibr.log"/>
  </Run>
  <Application class="roc::Readout">
    <DoCalibr value="2"/>
    <NumRocs value="3"/>
    <BufferSize value="65536"/>
    <NumBuffers value="100"/>
    <RawFile value="/d/cbm06/cbmdata/SEP08/raw/run028/run028*.lmd"/>
    <MbsServerKind value="Stream"/>
    <CalibrFile value="testcal.lmd"/>
    <MbsFileSizeLimit value="110"/>
  </Application>
 </Context>
</dabc>
\end{verbatim}


 
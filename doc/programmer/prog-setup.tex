[programmer/prog-setup.tex]
\label{prog_setup}
\section{Parameter class}
\label{prog_setup_parameter}
Configuration and status information of objects can be represented by the \class{Parameter} class.
Any object derived from \class{WorkingProcessor} class
(e.~g.~ \class{Application}, \class{Device}, \class{Module}, and \class{Port})
can have a list of parameters assigned to it.   
  
There are a number of \class{WorkingProcessor} methods to create parameter objects of different kinds and access their values. 
These are shown in the following table:

\begin{tabular}{|l|l|lll|}
   \hline
Type & Class  & Create & Getter & Setter \\
   \hline
string &  StrParameter    & CreateParStr()    & GetParStr()     & SetParStr ()    \\
double &  DoubleParameter & CreateParDouble() & GetParDouble()  & SetParDouble()  \\
int    &  IntParameter    & CreateParInt()    & GetParInt()     & SetParInt()     \\
bool   &  StrParameter    & CreateParBool()  & GetParBool()    & SetParBool()   \\
   \hline
\end{tabular}

The \func{CreatePar...()} methods will internally create a new
\class{Parameter} of the specified name if it does not exist before.
For any type of parameter the \func{GetParStr()} and \func{SetParStr()} methods can be used
which will deliver the parameter value as text string expression.

As one can see, to represent a boolean value a string parameter is used. 
If text of string is "true" (in lower case),
the boolean value is recognized as \keyw{true}, 
otherwise as \keyw{false}.

It is recommended to use these \class{WorkingProcessor} methods to create parameters and access their values;
but one can also use \func{FindPar()} method to find any parameter object 
and use its methods directly.   


\section{Use parameter for control}
\label{prog_setup_parametercontrol}
One advantage of the \dabc~ parameter objects is that parameter values can be 
observed and changed by a control system.

When a parameter value is changed in the program by a \func{SetPar...} method, 
the control system is informed and represents such change in an appropriate GUI element.
On the other hand, if the user modifies a parameter value in the GUI, 
the value of the parameter object will be changed and the corresponding parent object 
(\class{Module}, \class{Device}) gets a callback via 
virtual method \func{ParameterChanged()}. 
By implementing a suitable reaction in this call, 
one could reconfigure or adjust the running program on the fly.

A parameter object may be "fixed" via \func{Parameter::SetFixed()} method. 
This disables possibility to change the parameter value, both
from the program and the control/configuration system side. 
Only when the "fixed" flag is reset to \keyw{false},
the parameter can be modified again.

Not all parameters objects should be visible to the control system. 
Each parameter has a \strong{visibility flag} 
which is assigned to the parameter instance when it is created.
Only when \func{Parameter::IsVisible()} returns \keyw{true}, parameter will be  
known (visible) to the control system. Even if parameter is seen
from control system, it only can be changed \strong{from} control system 
when flag \func{Parameter::IsChangable()} returns \keyw{true}. 
   
Default flags values for newly created parameters can be set in
\func{WorkingProcessor::SetParDflts()} function.
For visibility user should specify level, which is compared with
global visibility level for parameters (aka debug level).
This global level can be changed by
\func{WorkingProcessor::SetGlobalParsVisibility()} static function or
in configuration file (value "Context/Run/parslevel"). Normally module parameters
has visibility level 1, module items (port, pool handle) parameters - 3, 
configuration parameters - 5. 
Thus, to see all parameters in control system, one should set "parslevel = 5".


\section{Example of parameters usage}
\label{prog_setup_parameterexample}
Let's consider an example of a module which uses parameters:

\begin{verbatim}
class UserModule : public dabc::ModuleAsync {
   public:
      UserModule(const char* name, dabc::Command* cmd = 0) : 
         dabc::ModuleAsync(name, cmd)
      {
         CreateParBool("Output", true);
         CreateParInt("Counter", 0);
         CreateTimer("Timer", 1.0, false);
      }
      
      virtual void ProcessTimerEvent(dabc::Timer*)
      {
         SetParInt("Counter", GetParInt("Counter")+1);
         if (GetParBool("Output")) 
            DOUT1(("Counter = %d", GetParInt("Counter")));
      }
}; 
\end{verbatim}

In the module constructor two parameters are created - boolean and integer, 
and a timer with $1\mbox{~s}$ period.
When the module is started, the value of integer parameter "Counter" 
will be changed every second.
If boolean parameter "Output" is set to \keyw{true}, 
the counter value will be displayed on debug output.

Using a control system, the value of the boolean parameter can be changed. 
To detect and react on such change,
one should implement following method: 
 
\begin{verbatim}
virtual void ParameterChanged(dabc::Parameter* par) 
{
   if (par->IsName("Output")) 
   DOUT1(("Output flag changed to %s", DBOOL(GetParBool("Output")));
}
\end{verbatim}

For performance reasons one should avoid to use parameter getter/setter methods 
(like  \func{GetParBool()} or \func{SetParInt()}) inside a loop 
being executed many times. The main purpose of a parameter
object is to provide a connection to the control and configuration system.
In other situations simple class members should be used.


\section{Configuration parameters}
\label{prog_setup_configurationparameter}
Another use case of parameters consists in the object configuration.
When one creates an object, like a module or a device, 
it is often necessary to deliver one or several configuration values 
to the constructor, e.~g.~ the required
number of input ports, or a server socket port number. 

For such situation configuration parameter are defined.
These parameters should be created and set 
in the object constructor with following methods only:


\bdes
\item[GetCfgStr]  string
\item[GetCfgDouble]   double 
\item[GetCfgInt]   integer
\item[GetCfgBool]   boolean 
\edes

All these methods have following arguments: the parameter name, 
a default value [optional], and a pointer to a \class{Command} object [optional].
Let's add one configuration parameter to our module constructor:

\begin{small}
\begin{verbatim}
UserModule(const char* name, dabc::Command* cmd = 0) : 
   dabc::ModuleAsync(name, cmd)
{
   CreateParBool("Output", true);
   CreateParInt("Counter", 0);
   double period = GetCfgDouble("Period", 1.0, cmd);
   CreateTimer("Timer", period, false);
}
\end{verbatim}
\end{small}

Here the period of the timer is set via configuration parameter "Period". 
How will its value  be defined? 
First of all, it will be checked if a parameter of that name exists in command
\func{cmd}. If not, the appropriate entry will be searched in the configuration file. 
If the configuration file also does not contain such parameter, 
the specified default value $1.0$  will be used.


\section{Configuration file example}
\label{prog_setup_configfile}
The configuration file is an XML file in a \dabc-specific format, 
which contains values for some or all configuration parameters of the system. 

Let's consider this simple but functional configuration file:

\begin{small}
\begin{verbatim}
<?xml version="1.0"?>
<dabc version="1">
  <Context host="localhost" name="Generator">
    <Run>
      <lib value="libDabcMbs.so"/>
      <func value="InitMbsGenerator"/>
    </Run>
    <Module name="Generator">
       <Port name="Output">
          <OutputQueueSize value="5"/>
          <MbsServerPort value="6000"/>
       </Port>
    </Module>
  </Context>
</dabc>
\end{verbatim}
\end{small}

This is an example XML file for an MBS generator, which produces 
MBS events and provides them to an {\em MBS transport} server. 
To run that example, just "run.sh test.xml" should be executed in a shell.
Other applications
(\dabc~ or {\em Go4}) can connect to that server and read generated mbs events.


\section{Basic syntax}
\label{prog_setup_configfile_syntax}
A \dabc~ configuration file should always contain <dabc> as root node. 
Inside the <dabc> node one or several <Context> nodes should exists.
Each <Context> node represents the {\em application context} which runs as
independent executable. 
Optionally the <dabc> node can have <Variables> and <Defaults> nodes, 
which are described further in the following sections  \ref{prog_setup_configfile_variables}
and \ref{prog_setup_configfile_defaults}


\section{Context}
\label{prog_setup_configfile_context}
A <Context> node can have two optional attributes:
\bdes
\item["host"] host name, where executable should run, default is "localhost"
\item["name"] application (manager), default is the host name.
\edes

Inside a <Context> node configuration parameters for modules, devices, memory pools are
contained.
In the example file one sees several parameters for the output port of 
the generator module.  


\section{Run arguments}
\label{prog_setup_configfile_run}
Usually a <Context> node has a <Run> subnode, where the user may define different parameters, relevant for running the \dabc~ executable:

\bdes
\item[lib] name of a library which should be loaded. Several libraries can be specified.
\item[func] name of a function which should be called to create modules. 
This is an alternative to instantiating a subclass of \class{dabc::Application} 
(compare section \ref{prog_plugin_applicaton})
\item[runfunc] function name to run some sequence of operations (start, stop, reconfigure) over application. Useful
for batch mode                 
\item[port] ssh port number of remote host
\item[user] account name to be used for ssh (login without password should be possible)
\item[init] init script, which should be called before dabc application starts
\item[test] test script, which is called when test sequence is run by run.sh script
\item[timeout] ssh timeout 
\item[debugger] argument to run with a debugger. Value should be like "gdb -x run.txt --args", where file run.txt should contain commands "r bt q".
\item[workdir] directory where \dabc~ executable should start
\item[debuglevel] level of debug output on console, default 1
\item[logfile] filename for log output, default none  
\item[loglevel] level of log output to file, default 2 
\item[DIM\_DNS\_NODE] node name of DIM dns server, used by DIM controls implementation 
\item[DIM\_DNS\_PORT] port number of DIM dns server, used by DIM controls implementation
\item[cpuinfo] instantiate \class{dabc::CpuInfoModule} to show CPU and memory usage information. 
Value must be >= 0. If 0, only two parameters are created, if 15 - several ratemeters will be created.    
\item[parslevel] level of pars visibility for control system, default 1 
\edes


\section{Variables}
\label{prog_setup_configfile_variables}
In the root node <dabc> one can insert a <Variables> node which may contain 
definitions of one or several variables. Once defined, 
such variables can be used in any place of the configuration file to set parameter values.
In this case the syntax to set a parameter is:

\begin{small}
\begin{verbatim}
    <ParameterName value="${VariableName}"/>
\end{verbatim}
\end{small}

It is allowed to define a variable as a combination of text with another variable, 
but neither arithmetic nor string operations are supported. 

Using variables, one can modify the example in the following way:

\begin{small}
\begin{verbatim}

<?xml version="1.0"?>
<dabc version="1">
  <Variables>
    <myname value="Generator"/> 
    <myport value="6010"/> 
  </Variables>
  <Context name="Mgr${myname}">
    <Run>
      <lib value="libDabcMbs.so"/>
      <func value="InitMbsGenerator"/>
    </Run>
    <Module name="${myname}">
       <SubeventSize value="32"/>
       <Port name="Output">
          <OutputQueueSize value="5"/>
          <MbsServerPort value="${myport}"/>
       </Port>
    </Module>
  </Context>
</dabc>
\end{verbatim}
\end{small}

Here context name and module name are set via {\tt myname} variable,
and mbs server socket port is set via {\tt myport} variable.

There are several variables which are predefined by the configuration system:

\bbul
\item DABCSYS - top directory of \dabc~ installation
\item DABCUSERDIR - user-specified directory
\item DABCWORKDIR - current working directory
\item DABCNUMNODES - number of <Context> nodes in configuration files
\item DABCNODEID - sequence number of current <Context> node in configuration file 
\ebul

Any shell environment variable 
is also available as variable in the configuration file to set parameter values. 


\section{Default values}
\label{prog_setup_configfile_defaults}
There are situations when one needs to set the same value to several similar parameters,
for instance the same queue length for all output ports in the module. 
One possible way is to use syntax as described above. 
The disadvantage of such approach is that one must expand the XML file
to set each queue length explicitely from the appropriate variable;
so in case of a big number of ports the file will be very long and 
confusing to the user.

Another possibility to set several parameters at once 
consists in \strong{wildcard rules} using "*" or "?" symbols.
These can be defined in a  <Defaults> node: 

\begin{small}
\begin{verbatim}
<?xml version="1.0"?>
<dabc version="1">
  <Variables>
    <myname value="Generator"/> 
    <myport value="6010"/> 
  </Variables>

  <Context name="Mgr${myname}">
    <Run>
      <lib value="libDabcMbs.so"/>
      <func value="InitMbsGenerator"/>
    </Run>
    <Module name="${myname}">
       <SubeventSize value="32"/>
       <Port name="Output">
          <MbsServerPort value="${myport}"/>
       </Port>
    </Module>
  </Context>
  <Defaults>
    <Module name="*">
       <Port name="Output*">
          <OutputQueueSize value="5"/>
       </Port>
    </Module>
  </Defaults>
</dabc>
\end{verbatim}
\end{small}

In this example for all ports which names begin with the string "Output", 
and which belong to any module, the output queue length will be 5. 
A wildcard rule of this form will be applied for 
all contexts of the configuration file, 
i.~e.~ by such rule we set the output queue length for all modules on all nodes. 
This allows to configure a big multi-node cluster with
a compact XML file.

Another possibility to set default value for some parameters - create
parameter with the same name in parent object. Here word \strong{create} 
is crutial - one should use \func{CreateParInt()} method in module constructor - 
it is not enough just put additional tag in xml file. For instance, one can
create parameter "MbsServerPort" in generator module and than 
MBS server transport, created for output port, will use that value for 
as default server port number.  


\section{Usage of commands for configuration}
\label{prog_setup_configuration_commands}
Let's consider the possibility to configure a module by means of the \class{Command} class.
Here the use case is that
an object (like a module) should be created with fixed parameters,
ignoring the values specified in the configuration file.

In our example one can modify \func{InitMbsGenerator()} function in the following way:
\begin{small}
\begin{verbatim}
extern "C" void InitMbsGenerator() 
{
  dabc::Command* cmd = new dabc::CmdCreateModule("mbs::GeneratorModule", 
                                                 "Generator");
  cmd->SetInt("SubeventSize", 128);
  if (!dabc::mgr()->Execute(cmd)) {
     EOUT(("Cannot create generator module"));
     exit(1);
  }
    
    ...
}
\end{verbatim}
\end{small}

Here one adds an additional parameter of name "SubeventSize" to
the \class{CmdCreateModule} object, 
which will set the MBS subevent size to 128. 
The generator module constructor will get the parameter value via method
\func{GetCfgInt()}, as described in section \ref{prog_setup_configurationparameter}.
Since the parameters of the passed \func{cmd} object will override all other settings here,
the value of the corresponding <SubeventSize> entry in the configuration file has no effect.


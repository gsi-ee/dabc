[programmer/prog-plugin.tex]
\section{Introduction}
%\textit{put some general words on the plugin philosophy here}
A multi purpose DAQ system like \dabc~ requires to develop user specific code and adopt
this into the general framework. A common object oriented technique to realize such
extensability consists in the definition of base classes as interfaces for dedicated purposes.
The programmer may implement subclasses for these interfaces as \strong{Plug-Ins}
with the extended functionality that matches the data format, hardware, or other boundary conditions of the
data-taking experiment. Moreover, the  \dabc~ core itself applies such powerful plug-in mechanism to provide generic services in a flexible and maintenable manner.   

This chapter gives a brief description of all interface classes for the data acquisition 
processing itself. This covers the processing \strong{Modules}, the \strong{Transport} and 
\strong{Device} objects that move data between the DAQ components, 
and the \strong{Application} that is responsible for the node set-up and run control.
A \strong{Factory} pattern is used to introduce new classes to the framework and let them
be available by name at runtime.

\section{Modules}
\subsection{ModuleSync}
The data processing functionality is usually implemented by subclassing 
   the \class{dabc::ModuleSync} base class. 

\begin{compactenum}
\item  The constructor of \class{dabc::ModuleSync} subclass  creates all 
      ports, may initialize the pool handles, and may 
      define commands and parameters. 
\item  The virtual \func{ExecuteCommand()} method of \class{dabc::ModuleSync} 
      subclass may implement the callbacks of the defined commands.
\item  The virtual \func{MainLoop()} method of \class{dabc::ModuleSync} 
      subclass implements the processing job.
\item  The user code shall not directly access the 
      memory pools to request new buffers. Instead, it can use methods of 
      \class{dabc::Module} with a \class{dabc::PoolHandle} object as argument. 
      These methods may block the \func{MainLoop()}.
\item  The user code can send and receive buffers 
      from and to ports by methods of \class{dabc::ModuleSync}. 
      These methods may block the \func{MainLoop()}.
\end{compactenum}

\subsection{Special modules}
For special set ups (e.g. Bnet), the framework provides 
   \class{dabc::Module} subclasses with generic functionality 
   (e.g. \class{bnet::BuilderModule}, \class{bnet::FilterModule}). 
   In this case, the user specific parts like data formats are 
   implemented by subclassing these special module classes.

   
\begin{compactenum}

\item  Instead of implementing \func{MainLoop()}, other virtual 
      methods (e.g. \func{DoBuildEvent()}, \func{TestBuffer()}) may be 
      implemented that are implicitly called by the superclass \func{MainLoop()}.
\item  The special base classes may provide additional 
      methods to be used for data processing.    
\end{compactenum}

\section{Device and transport}
All data transport functionality is implemented by 
   subclassing  \class{dabc::Device} and \class{dabc::Transport} base classes. 
\begin{compactenum}
\item  The \class{dabc::Device} subclass constructor may create  
      pool handles and may define commands and parameters. 
\item  The virtual \func{ExecuteCommand()} method of \class{dabc::Device} 
      subclass may implement the callbacks of the defined commands.
\item  Factory method \func{CreateTransport()} of \class{dabc::Device}
      subclass defines which transport instance is created 
      by the framework whenever this device shall be connected to 
      the port of a module.   
\item  The virtual \func{Send()} and \func{Recv()} methods of the 
      \class{dabc::Transport} subclass must implement the actual transport 
      from and to a connected port, respectively.
\item  To implement a simple user defined transport 
      (e.g. read from a special socket protocol), 
      there is another base class \class{dabc::DataTransport} 
      (subclass of\class{dabc::Transport}). 
      This class already provides queues for (optionally)
      input and output buffers and a data backpressure mechanism. 
      Instead of \func{Send} and \func{Recv()}, the user subclass must implement 
      special virtual methods, e.g. \func{ReadBegin()}, \func{ReadComplete(buffer)} 
      to request the next buffer of a given size from the base class, 
      and to peform filling this buffer from the associated user device, respectively. 
      These methods are called implicitely from the framework at the right time. 
\end{compactenum}

\section{Factories}
The set up of the application specific module and device objects is done 
   by \class{dabc::Factory} subclasses.
\begin{compactenum}
\item  All factories are registered and kept in the global manager. 
      The access to the factories' functionality is done via methods of the manager. 
\item  All factories are instantiated as static singletons when 
      loading the libraries that defines them. 
      The user need not to create them explicitely.       
\item  The \class{dabc::Factory} subclasses must implement methods \func{CreateModule(classname)}, 
      and \func{CreateDevice(classname)} , to instantiate modules or 
      devices, respectively. The user must subclass 
      the \class{dabc::Factory} to add own classes to the system. \strong{Note:} the factory 
      method for transport objects is not in this factory, but in the 
      corresponding \class{dabc::Device} subclass. 
\item  The manager can keep more than one factory and scans 
      all factories to produce the requested object class. 
\item  The framework provides several factories for predefined 
      implementations (e.g. \class{bnet::SenderModule}, \class{dabc::VerbsDevice}).      
\end{compactenum}

\section{The DABC Application}
The specific application controlling code is defined in 
   the \class{dabc::Application}.   
\begin{compactenum}
\item  The manager has exactly one application object. 
      The user must provide a \class{dabc::Application} subclass.
\item  The \class{dabc::Application} is instantiated and registered on 
      startup time by a global \func{InitPlugins()} function. This function 
      is declared and executed in the framework, but defined in the user 
      library and thus knows the user specific plug-in classes. 
      This trick is necessary to decouple the executable main function 
      (e.g. \xdaq executive) from the user specific part.  
\item  The user may implement virtual methods \func{UserConfigure()} ,  
      \func{UserEnable()}, \func{UserBeforeStart()}, 
      \func{UserAfterStop()}, \func{UserHalt()} in his/her 
      \class{dabc::Application} subclass. These methods are executed by the 
      framework state machine before or after the corresponding state 
      transitions to do additional application specific configuration, 
      run control, and clean-up actions. Note: all generic state machine 
      actions (e.g. cleanup of modules and devices, starting 
      and stopping the working processors) are already handled by 
      the framework at the right time and need not to be invoked explicitely here.
\item  The application  may register parameters that 
      define the application's configuration. These parameters can be set at 
      runtime from the configuration and controls system.
\item  The methods of the application  should use 
      application factories to create modules and 
      devices by name string. However, for convenience the user may 
      implement factory methods \func{CreateModule()} and \func{CreateDevice()} straight 
      in his/her \class{dabc::Application} subclass.
\item   For special DAQ topologies (e.g. Bnet), the framework offers 
      implementations of the \class{dabc::Application} containing the 
      generic functionality (e.g. \class{bnet::WorkerApplication}, \class{bnet::ClusterApplication}). 
      In this case, the user specific parts are implemented by subclassing 
      these and implementing additional virtual methods (e.g. \func{CreateReadout()}).    
\end{compactenum}







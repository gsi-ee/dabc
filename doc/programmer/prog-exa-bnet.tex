[programmer/prog-exa-bnet.tex]

\section{Overview}

In complex experiments there are lot of front-end systems, which runs in parallel.
They takes data and marks them with trigger information or just with time stamps.
To be able analyze such data, all portions belonging to the same event (or time stamp),   
should be combined in one processing unit. Such task usually called event building.

To support such functionality in DABC, special sub-framework of building network (BNET) was introduced. 
It's main aim - simplify implementation of event building for user-specific data.
 



\section{Controller and workers}

Event building task usually distributed over several nodes, which should be controlled.
Therefore in BNET all nodes classified by their functionality on two kinds: controller and
workers. Workers perform all kinds of data transport and analysis codes while controller 
configures and steers all workers. 

Functionality of controller is implemented in \class{bnet::ClusterApplication} class.
Via controlling interface cluster controller distributes commands, coming from user,
to all workers, observes status of all workers and reconfigures them when errors on
some workers are detected.
 
Basic functionality of worker implemented in \class{bnet::WorkerApplication} class.
Its main aim - by commands from cluster controller instantiate all necessary modules, 
configure and connect them together. While some of the modules should be user-specific,
class provides virtual methods, which should be implemented in user-specific part. 


\section{Writing user classes}


\section{BNET-test example}


\section{BNET for MBS implementation}

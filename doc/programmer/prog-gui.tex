[programmer/prog-gui.tex]
\section{GUI Guide lines}
\section{GUI Panels}
\subsection{\dabc~ launch panel}
\subsection{\mbs~ launch panel}
\subsection{Combined \dabc~ and \mbs~ launch panel}
\subsection{Command panel}
\subsection{Parameter selection panel}
\subsection{Monitoring panels}
\subsection{Parameter table}
\subsection{Logging window}
\section{GUI save/restore setups}
\section{Application specific GUI plug-in}
\subsection{Application interfaces}
Besides the generic part of the GUI it might be useful to have specific user panels as well, integrated in the generic GUI. This is provided by interface classes. A user may implement these interfaces in his own menues. He can connect his own call back functions to parameters, and a command function to be called when a command shall be executed. He may create his own panels for display using the graphical primitives like rate meters.

\subsection{Java Interfaces to be implemented by application}
\subsubsection{Interface xiUserPanel}
{\tt public abstract void init(xiDesktop desktop, ActionListener actionlistener);}
Called by xgui after instantiation. The desktop can be used to add frames (see below).\\
{\tt public String getHeader();}
Must return a header/name text after instantiation.\\
{\tt public String getToolTip();}
Must return a tooltip text after instantiation.\\
{\tt public ImageIcon getIcon();}
Must return an icon after instantiation.\\
{\tt public xiUserCommand getUserCommand();}
Must return an object implementing xiUserCommand, or null. See below.\\
{\tt public void setDimServices(xiDimBrowser browser);}
Called by xgui whenever the DIM services had been changed.
The browser provides the command and parameter list (see below). One can select and store references to commands or parameters. A xiUserInfoHandler can be registered for each selected parameter. Then the infoHandler method is called for each parameter update.\\
{\tt public void releaseDimServices();}
All local references to commands or parameters must be cleared!

\subsubsection{public interface xiUserCommand}
{\tt public boolean getArgumentStyleXml(String scope, String command);}
Return true if command shall be composed as XML string, false if MBS style string. scope is specified in the XML command descriptor, command is the full command name.

\subsubsection{public interface xiUserInfoHandler}
{\tt public void infoHandler(xiDimParameter p);}

\subsection{Java Interfaces provided by xgui}
\subsubsection{Interface xiDesktop}
{\tt public void addDesktop(JInternalFrame frame, String name);}

\subsubsection{Interface xiDimBrowser}
\begin{verbatim}
public xiDimParameter[] getParameters();
public xiDimCommand[] getCommands();
public void setInfoHandler(xiDimParameter parameter, 
                           xiUserInfoHandler infohandler);
public void sleep(int s);
\end{verbatim}

\subsubsection{Interface xiDimCommand}
\begin{verbatim}
public void exec(String command);
public xiParser getParserInfo();
\end{verbatim}

\subsubsection{Interface xiDimParameter}
\begin{verbatim}
public xRecordMeter getMeter();
public xRecordState getState();
public xRecordInfo getInfo();
public xiParser getParserInfo();
\end{verbatim}

\subsubsection{Interface xiParser}
\begin{verbatim}
public String getDns();
public String getNode();
public String getNodeName();
public String getNodeID();
public String getApplicationFull();
public String getApplication();
public String getApplicationName();
public String getApplicationID();
public String getName();
public String getNameSpace();
public String[] getItems();
public String getFull();
public String getFull(boolean build);
public String getCommand();
public String getCommand(boolean build);
public int getType();
public int getState();
public int getVisibility();
public int getMode();
public int getQuality();
public int getNofTypes();
public int[] getTypeSizes();
public String[] getTypeList();
public String getFormat();
public boolean isNotSpecified();
public boolean isSuccess();
public boolean isInformation();
public boolean isWarning();
public boolean isError();
public boolean isFatal();
public boolean isAtomic();
public boolean isGeneric();
public boolean isState();
public boolean isInfo();
public boolean isRate();
public boolean isHistogram();
public boolean isCommandDescriptor();
public boolean isHidden();
public boolean isVisible();
public boolean isMonitor();
public boolean isChangable();
public boolean isImportant();
public boolean isLogging();
public boolean isArray();
public boolean isFloat();
public boolean isDouble();
public boolean isInt();
public boolean isLong();
public boolean isChar();
public boolean isStruct();
\end{verbatim}

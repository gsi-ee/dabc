\section{Generic Java GUI}
A generic GUI has been implemented in Java. It processes all services found on the DIM name server. According the description above it does the following:
\begin{compactitem}[$\bullet$]
\item Get list of commands and parameters and create objects for each
\item Put parameters in a table
\item Put commands in a command tree
\item Create graphics panels for rate meters, states, histograms, and infos
\end{compactitem}
\subsection{Application servers}
Any application which can implement DIM services can be controlled by the generic GUI if it follows the protocol describesd above. The first application was DABC, the second one MBS.

\subsection{Application interfaces}
Besides the generic part of the GUI it might be useful to have specific user panels as well, integrated in the generic GUI. This is provided by interface classes. A user may implement these interfaces in his own menues. He can connect his own call back functions to parameters, and a command function to be called when a command shall be executed. He may create his own panels for display using the graphical primitives like rate meters.

\subsection{Java Interfaces to be implemented by application}
\subsubsection{Interface xiUserPanel}
{\tt public abstract void init(xiDesktop desktop, ActionListener actionlistener);}
Called by xgui after instantiation. The desktop can be used to add frames (see below).\\
{\tt public String getHeader();}
Must return a header/name text after instantiation.\\
{\tt public String getToolTip();}
Must return a tooltip text after instantiation.\\
{\tt public ImageIcon getIcon();}
Must return an icon after instantiation.\\
{\tt public xiUserCommand getUserCommand();}
Must return an object implementing xiUserCommand, or null. See below.\\
{\tt public void setDimServices(xiDimBrowser browser);}
Called by xgui whenever the DIM services had been changed.
The browser provides the command and parameter list (see below). One can select and store references to commands or parameters. A xiUserInfoHandler can be registered for each selected parameter. Then the infoHandler method is called for each parameter update.\\
{\tt public void releaseDimServices();}
All local references to commands or parameters must be cleared!

\subsubsection{public interface xiUserCommand}
{\tt public boolean getArgumentStyleXml(String scope, String command);}
Return true if command shall be composed as XML string, false if MBS style string. scope is specified in the XML command descriptor, command is the full command name.

\subsubsection{public interface xiUserInfoHandler}
{\tt public void infoHandler(xiDimParameter p);}

\subsection{Java Interfaces provided by xgui}
\subsubsection{Interface xiDesktop}
{\tt public void addDesktop(JInternalFrame frame, String name);}

\subsubsection{Interface xiDimBrowser}
\begin{verbatim}
public xiDimParameter[] getParameters();
public xiDimCommand[] getCommands();
public void setInfoHandler(xiDimParameter parameter, 
                           xiUserInfoHandler infohandler);
public void sleep(int s);
\end{verbatim}

\subsubsection{Interface xiDimCommand}
\begin{verbatim}
public void exec(String command);
public xiParser getParserInfo();
\end{verbatim}

\subsubsection{Interface xiDimParameter}
\begin{verbatim}
public xRecordMeter getMeter();
public xRecordState getState();
public xRecordInfo getInfo();
public xiParser getParserInfo();
\end{verbatim}

\subsubsection{Interface xiParser}
\begin{verbatim}
public String getDns();
public String getNode();
public String getNodeName();
public String getNodeID();
public String getApplicationFull();
public String getApplication();
public String getApplicationName();
public String getApplicationID();
public String getName();
public String getNameSpace();
public String[] getItems();
public String getFull();
public String getFull(boolean build);
public String getCommand();
public String getCommand(boolean build);
public int getType();
public int getState();
public int getVisibility();
public int getMode();
public int getQuality();
public int getNofTypes();
public int[] getTypeSizes();
public String[] getTypeList();
public String getFormat();
public boolean isNotSpecified();
public boolean isSuccess();
public boolean isInformation();
public boolean isWarning();
public boolean isError();
public boolean isFatal();
public boolean isAtomic();
public boolean isGeneric();
public boolean isState();
public boolean isInfo();
public boolean isRate();
public boolean isHistogram();
public boolean isCommandDescriptor();
public boolean isHidden();
public boolean isVisible();
public boolean isMonitor();
public boolean isChangable();
public boolean isImportant();
public boolean isLogging();
public boolean isArray();
public boolean isFloat();
public boolean isDouble();
public boolean isInt();
public boolean isLong();
public boolean isChar();
public boolean isStruct();
\end{verbatim}
\subsection{Setup files}
\subsubsection{Records}
File {\tt Records.xml}
\begin{verbatim}
<?xml version="1.0" encoding="utf-8"?>
<Record>
<Meter name="DABC/X86-7/MSG/DataRateKb" 
       visible="true" 
       mode="0" 
       auto="false" 
       log="false" 
       low="00000000.0" 
       up="00016000.0" 
       color="Red"/>
</Record>
\end{verbatim}
\subsubsection{Parameter filter}
File {\tt Selection.xml}
\begin{verbatim}
<?xml version="1.0" encoding="utf-8"?>
<Selection>
<Full contains="Date" filter="false" />
<Node contains="X86-7" filter="false" />
<Application contains="MSG" filter="false" />
<Name contains="*" filter="false" />
<Records Only="true"  Rates="true"  States="false"  Infos="false" />
</Selection>
\end{verbatim}
\subsubsection{Windows layout}
File {\tt Layout.xml}
\begin{verbatim}
<?xml version="1.0" encoding="utf-8"?>
<Layout>
<WindowLayout>
<Main shape="357,53,857,953" columns="0" show="true"/>
<Command shape="0,230,650,200" columns="0" show="false"/>
<Parameter shape="20,259,578,386" columns="0" show="false"/>
<Logger shape="0,650,680,150" columns="0" show="false"/>
<Meter shape="463,13,413,236" columns="4" show="false"/>
<State shape="85,504,313,206" columns="2" show="false"/>
<Info shape="521,482,613,217" columns="1" show="false"/>
<Histogram shape="124,508,613,206" columns="3" show="false"/>
<DabcLauncher shape="0,0,100,100" columns="0" show="false"/>
<MbsLauncher shape="50,14,404,272" columns="0" show="false"/>
<DabcMbsLauncher shape="0,0,430,424" columns="0" show="false"/>
<ParameterSelect shape="300,0,271,326" columns="0" show="true"/>
<ParameterList shape="13,364,810,426" columns="1" show="true"/>
</WindowLayout>
<TableLayout>
<Parameter width="74,74,74,74,74,74,74,74" />
</TableLayout>
</Layout>
\end{verbatim}
\subsubsection{\dabc~ launch panel values}
File {\tt DabcLaunch.xml}
\begin{verbatim}
<?xml version="1.0" encoding="utf-8"?>
<DabcLaunch>
<Labels
DabcMaster="DABC Master"
DabcName="DABC Name"
DabcUserPath="DABC user path"
DabcSystemPath="DABC system path"
DabcSetup="DABC setup file"
DabcScript="DABC Script"
DabcServers="%Number of needed DIM servers%"
/>
<Fields
DabcMaster="lxg0523.gsi.de"
DabcName="Controller:41"
DabcUserPath="/misc/goofy/dabc/work"
DabcSystemPath="/misc/goofy/sniff/dabc"
DabcSetup="SetupMbs.xml"
DabcScript="ps"
DabcServers="5"
/>
</DabcLaunch>
\end{verbatim}
\subsubsection{\mbs~ launch panel}
File {\tt MbsLaunch.xml}
\begin{verbatim}
<?xml version="1.0" encoding="utf-8"?>
<MbsLaunch>
<Labels
MbsMaster="MBS Master"
MbsUserPath="MBS User path"
MbsSystemPath="MBS system path"
MbsScript="MBS Script"
MbsCommand="Script command"
MbsServers="%Number of needed DIM servers%"
/>
<Fields
MbsMaster="x86g-4"
MbsUserPath="v50/x86/newmbs"
MbsSystemPath="/daq/usr/goofy/mbswork/v51"
MbsScript="script/remote_exe.sc"
MbsCommand="m_rising v50/x86/newmbs . x86g-4 x86-7"
MbsServers="3"
/>
</MbsLaunch>
\end{verbatim}





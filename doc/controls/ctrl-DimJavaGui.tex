[controls/ctrl-DimJavaGui.tex]
\section{Generic Java GUI}
A generic GUI has been implemented in Java. It processes all services found on the DIM name server. According the description above it does the following:
\begin{compactitem}[$\bullet$]
\item Get list of commands and parameters and create objects for each
\item Put parameters in a table
\item Put commands in a command tree
\item Create graphics panels for rate meters, states, histograms, and infos
\end{compactitem}
\subsection{Application servers}
Any application which can implement DIM services can be controlled by the generic GUI if it follows the protocol describesd above. The first application was DABC, the second one MBS.

\subsection{Application interfaces}
Besides the generic part of the GUI it might be useful to have specific user panels as well, integrated in the generic GUI. This is provided by interface classes. A user may implement these interfaces in his own menues. He can connect his own call back functions to parameters, and a command function to be called when a command shall be executed. He may create his own panels for display using the graphical primitives like rate meters.

\section{Setup files}
\subsection{Records}
File {\tt Records.xml}
\begin{verbatim}
<?xml version="1.0" encoding="utf-8"?>
<Record>
<Meter name="DABC/X86-7/MSG/DataRateKb" 
       visible="true" 
       mode="0" 
       auto="false" 
       log="false" 
       low="00000000.0" 
       up="00016000.0" 
       color="Red"/>
</Record>
\end{verbatim}
\subsection{Parameter filter}
File {\tt Selection.xml}
\begin{verbatim}
<?xml version="1.0" encoding="utf-8"?>
<Selection>
<Full contains="Date" filter="false" />
<Node contains="X86-7" filter="false" />
<Application contains="MSG" filter="false" />
<Name contains="*" filter="false" />
<Records Only="true"  Rates="true"  States="false"  Infos="false" />
</Selection>
\end{verbatim}
\subsection{Windows layout}
File {\tt Layout.xml}
\begin{verbatim}
<?xml version="1.0" encoding="utf-8"?>
<Layout>
<WindowLayout>
<Main shape="357,53,857,953" columns="0" show="true"/>
<Command shape="0,230,650,200" columns="0" show="false"/>
<Parameter shape="20,259,578,386" columns="0" show="false"/>
<Logger shape="0,650,680,150" columns="0" show="false"/>
<Meter shape="463,13,413,236" columns="4" show="false"/>
<State shape="85,504,313,206" columns="2" show="false"/>
<Info shape="521,482,613,217" columns="1" show="false"/>
<Histogram shape="124,508,613,206" columns="3" show="false"/>
<DabcLauncher shape="0,0,100,100" columns="0" show="false"/>
<MbsLauncher shape="50,14,404,272" columns="0" show="false"/>
<DabcMbsLauncher shape="0,0,430,424" columns="0" show="false"/>
<ParameterSelect shape="300,0,271,326" columns="0" show="true"/>
<ParameterList shape="13,364,810,426" columns="1" show="true"/>
</WindowLayout>
<TableLayout>
<Parameter width="74,74,74,74,74,74,74,74" />
</TableLayout>
</Layout>
\end{verbatim}
\subsection{\dabc~ launch panel values}
File {\tt DabcLaunch.xml}
\begin{verbatim}
<?xml version="1.0" encoding="utf-8"?>
<DabcLaunch>
<Labels
DabcMaster="DABC Master"
DabcName="DABC Name"
DabcUserPath="DABC user path"
DabcSystemPath="DABC system path"
DabcSetup="DABC setup file"
DabcScript="DABC Script"
DabcServers="%Number of needed DIM servers%"
/>
<Fields
DabcMaster="lxg0523.gsi.de"
DabcName="Controller:41"
DabcUserPath="/misc/goofy/dabc/work"
DabcSystemPath="/misc/goofy/sniff/dabc"
DabcSetup="SetupMbs.xml"
DabcScript="ps"
DabcServers="5"
/>
</DabcLaunch>
\end{verbatim}
\subsection{\mbs~ launch panel}
File {\tt MbsLaunch.xml}
\begin{verbatim}
<?xml version="1.0" encoding="utf-8"?>
<MbsLaunch>
<Labels
MbsMaster="MBS Master"
MbsUserPath="MBS User path"
MbsSystemPath="MBS system path"
MbsScript="MBS Script"
MbsCommand="Script command"
MbsServers="%Number of needed DIM servers%"
/>
<Fields
MbsMaster="x86g-4"
MbsUserPath="v50/x86/newmbs"
MbsSystemPath="/daq/usr/goofy/mbswork/v51"
MbsScript="script/remote_exe.sc"
MbsCommand="m_rising v50/x86/newmbs . x86g-4 x86-7"
MbsServers="3"
/>
</MbsLaunch>
\end{verbatim}

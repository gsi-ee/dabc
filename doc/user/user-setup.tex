[user/user-setup.tex]
\lsection[Installing DABC]{user-install}{Installing \dabc}
\index{DABC!Installation}
When working at the gsi linux cluster, the \dabc~ framework is already installed and will
be maintained by people of the gsi EE department. Here \dabc~ needs just to be
activated from any gsi shell by typing {\tt . dabclogin}. In this case, 
please skip this installation section and proceed with following section \ref{user-env} describing
the set-up of the user environment.

However, if working on a separate DAQ cluster outside gsi, 
it is mandatory to install the \dabc~ software
from scratch. 
Hence the \dabc~ distribution is available for download at \hyperref{http://dabc.gsi.de}{}{}{http://dabc.gsi.de}.
It is provided as a compressed tarball of sources {\tt dabc1.tar.gz}.
The following steps describe the recommended installation procedure:

\bnum
\item {\bf Unpack this \dabc~ distribution} at an appropriate installation directory,
e.~g.~:
\begin{verbatim}
cd /opt/dabc; 
tar zxvf dabc1.tar.gz
\end{verbatim}
This will extract the archive into a subdirectory which is labelled
with the current version number like {\tt /opt/dabc/dabc\_1\_0.00}.
This becomes the future \dabc~ system directory.

\item {\bf Prepare the \dabc~ environment login script}:
A template for this script can be found at {\tt scripts/dabclogin.sh}
\bbul
  \item Edit the {\tt DABCSYS} environment according to your local installation directory. 
  This is done in the following lines:
  \begin{verbatim}
  export DABCSYS=/opt/dabc\_installations/dabc_1_0.00  
  \end{verbatim}  
  
  \item Edit the {\tt DIM\_DNS\_NODE} environment according to the machine where 
  the DIM name server \cite{DIM} for the \dabc~ control system will run:
   \begin{verbatim}
  export DIM_DNS_NODE=hostname.domain.de
  \end{verbatim}  
  \item Copy the script to a location in your global {\tt \$PATH} for later login,
  e.~g.~ {\tt /usr/bin}. Alternatively, you
  may set an \func{alias} to the full pathname of {\tt dabclogin.sh} in your shell profile.
\ebul

\item Execute the just modified login script in your shell to set the environment:  
  \begin{verbatim}
  . dabclogin.sh
  \end{verbatim} 
  This will set the environment for the compilation.

\item Change to the \dabc~ installation directory and start the build:
  \begin{verbatim}
  cd \$DABCSYS
  make
  \end{verbatim} 
  This will compile the \dabc~ framework and install a suitable version of DIM in a
  subdirectory of {\tt \$DABCSYS/dim}.

\enum

After succesful compilation, the \dabc~ framework installation is complete
and can be used from any shell after invoking {\tt . dabclogin.sh}
The next sections \ref{user-env} and \ref{user-setup} will describe further steps 
to set-up the \dabc~ working environment for each user.


\lsection[Set-up the DABC environment]{user-env}{Set-up the  \dabc~ environment}
\index{DABC!Environment set-up}
Once the general \dabc~ framework is installed on a system, still each user
must "activate" the environment and do further preparations to work with it.

\bnum
\item Execute the \dabc~ login script in a linux shell to set the environment.
At gsi linux installation, this is done by  
\begin{verbatim}
  . dabclogin
  \end{verbatim} 
For the user installation as described in above section \ref{user-install},
by default the script is named   
  \begin{verbatim}
  . dabclogin.sh
  \end{verbatim} 
The login script will already enable the \dabc~ framework for
compilation of user written components. Additionally, 
the general executable {\tt dabc\_run} now provides
the \dabc~ runtime environment and may be started directly 
for simple "batch mode" applications on a single node. 

However, further preparations are necessary if \dabc~ shall be used with
DIM control system and GUI.

\item Open a dedicated shell on the machine that shall provide the DIM name server,
e.~g.~ {\tt ssh nsnode.cluster.domain}
Then call 
\begin{verbatim}
. dabclogin.sh
dimDns  
\end{verbatim} 
to launch the DIM name server. This is done \strong{once} at the beginning of
the DAQ setup; usually the DIM name server needs not to be shut down 
when \dabc~ applications terminate.

\item Set the DIM name server environment variable in any \dabc~ working shell (e.~g.~
the shell that will start the dabc gui later):
\begin{verbatim}
. dabclogin.sh
export DIM\_DNS\_NODE=nsnode.cluster.domain
\end{verbatim} 
Note that a user installation of \dabc~ framework may set this 
environment variable already in the general {\tt dabclogin.sh} script, if the DIM name server will 
mostly run for all users on the same default node (see section \ref{user-install}).  

\item Now the \dabc~ GUI can be started in such prepared shell by typing {\tt dabc}, (or 
{\tt mbs} for a plain \mbs~ gui, resp.). See below in gui section.  

\enum


\lsection[Setting up DABC user workspace]{user-setup}{Setting up \dabc\ user workspace}
\index{TODO!Setup}
To operate a \dabc\ application one should create a dedicated 
working directory to keep all relevant files:
\bbul
\item Setup files for \dabc\ (XML).
\item Log files (text).
\ebul
The GUI may run on a different directory with no access to the \dabc\ working directory.
There we place:
\bbul
\item Data files for startup panels (XML).
\item Configuration files for GUI (XML).
\ebul
Of cause, one can use one directory for all.
The setup files are application specific. Therefore they are described in
the application sections.
\subsection[DABC data simulator]{\dabc\ data simulator}
%\paref{user:Bnet}
\subsection[DABC MBS event server]{\dabc\ \mbs\ event server}
\paref{user:mbs}
\subsection{PCI connected front-ends}
%\paref{user:roc}


\lsection[Installation of additional plug-ins]{user-plugins}{Installation of additional plug-ins}
\index{TODO!Plugin installation}
example abb here to explain what to do. 

\lsubsection[Add plug-in packages to \$DABCSYS]{user-plugins-dabcsys}{Add plug-in packages to \$DABCSYS}
Describe here what to do if you have dabcsys installation priviliges.

\lsubsection[Plug-in packages in user directory]{user-plugins-userdir}{Plug-in packages in user directory}
How to use plug-in packages outside dabcsys. modify the library load paths in example
[user/user-setup.tex]
\lsection[Installing DABC]{user-install}{Installing \dabc}
\index{DABC!Installation}
When working at the gsi linux cluster, the \dabc~ framework is already installed and will
be maintained by people of the gsi EE department. Here \dabc~ needs just to be
activated from any gsi shell by typing {\tt . dabclogin}. In this case, 
please skip this installation section and proceed with following section \ref{user-env} describing
the set-up of the user environment.

However, if working on a separate DAQ cluster outside gsi, 
it is mandatory to install the \dabc~ software
from scratch. 
Hence the \dabc~ distribution is available for download at \hyperref{http://dabc.gsi.de}{}{}{http://dabc.gsi.de}.
It is provided as a compressed tarball of sources {\tt dabc1.tar.gz}.
The following steps describe the recommended installation procedure:

\bnum
\item {\bf Unpack this \dabc~ distribution} at an appropriate installation directory,
e.~g.~:
\begin{verbatim}
cd /opt/dabc; 
tar zxvf dabc1.tar.gz
\end{verbatim}
This will extract the archive into a subdirectory which is labelled
with the current version number like {\tt /opt/dabc/dabc\_1\_0.00}.
This becomes the future \dabc~ system directory.

\item {\bf Prepare the \dabc~ environment login script}:
A template for this script can be found at  
\begin{verbatim}
scripts/dabclogin.sh
\end{verbatim}


\bbul
  \item Edit the {\tt DABCSYS} environment according to your local installation directory. 
  This is done in the following lines:
  \begin{verbatim}
  export DABCSYS=/opt/dabc\_installations/dabc_1_0.00  
  \end{verbatim}  
  
  \item Edit the {\tt DIM\_DNS\_NODE} environment according to the machine where 
  the DIM name server \cite{DIM} for the \dabc~ control system will run:
   \begin{verbatim}
  export DIM_DNS_NODE=hostname.domain.de
  \end{verbatim}  
  \item Copy the script to a location in your global {\tt \$PATH} for later login,
  e.~g.~ {\tt /usr/bin}. Alternatively, you
  may set an \func{alias} to the full pathname of {\tt dabclogin.sh} in your shell profile.
\ebul

\item Execute the just modified login script in your shell to set the environment:  
  \begin{verbatim}
  . dabclogin.sh
  \end{verbatim} 
  This will set the environment for the compilation.

\item Change to the \dabc~ installation directory and start the build:
  \begin{verbatim}
  cd \$DABCSYS
  make
  \end{verbatim} 
  This will compile the \dabc~ framework and install a suitable version of DIM in a
  subdirectory of {\tt \$DABCSYS/dim}.

\enum

After succesful compilation, the \dabc~ framework installation is complete
and can be used from any shell after invoking {\tt . dabclogin.sh}
The next sections \ref{user-env} and \ref{user-setup} will describe further steps 
to set-up the \dabc~ working environment for each user.


\lsection[Set-up the DABC environment]{user-env}{Set-up the  \dabc~ environment}
\index{DABC!Environment set-up}
Once the general \dabc~ framework is installed on a system, still each user
must "activate" the environment and do further preparations to work with it.

\bnum
\item Execute the \dabc~ login script in a linux shell to set the environment.
At gsi linux installation, this is done by  
\begin{verbatim}
  . dabclogin
  \end{verbatim} 
For the user installation as described in above section \ref{user-install},
by default the script is named   
  \begin{verbatim}
  . dabclogin.sh
  \end{verbatim} 
The login script will already enable the \dabc~ framework for
compilation of user written components. Additionally, 
the general executable {\tt dabc\_run} now provides
the \dabc~ runtime environment and may be started directly 
for simple "batch mode" applications on a single node. 

However, further preparations are necessary if \dabc~ shall be used with
DIM control system and GUI.

\item Open a dedicated shell on the machine that shall provide the DIM name server,
e.~g.~ 
\begin{verbatim}
ssh nsnode.cluster.domain
\end{verbatim}
Then call 
\begin{verbatim}
. dabclogin.sh
dimDns  
\end{verbatim} 
to launch the DIM name server. This is done \strong{once} at the beginning of
the DAQ setup; usually the DIM name server needs not to be shut down 
when \dabc~ applications terminate.

\item Set the DIM name server environment variable in any \dabc~ working shell (e.~g.~
the shell that will start the dabc gui later):
\begin{verbatim}
. dabclogin.sh
export DIM\_DNS\_NODE=nsnode.cluster.domain
\end{verbatim} 
Note that a user installation of \dabc~ framework may set this 
environment variable already in the general {\tt dabclogin.sh} script, if the DIM name server will 
mostly run for all users on the same default node (see section \ref{user-install}).  

\item Now the \dabc~ GUI can be started in such prepared shell by typing {\tt dabc}, (or 
{\tt mbs} for a plain \mbs~ gui, resp.). See below in gui section.  

\enum


\lsection[Setting up DABC user workspace]{user-setup}{Setting up \dabc\ user workspace}
\index{TODO!Setup}
To operate a \dabc\ application one should create a dedicated 
working directory to keep all relevant files:
\bbul
\item Setup files for \dabc\ (XML).
\item Log files (text).
\ebul
The GUI may run on a machine with no access to the \dabc\ working directory,
e.~g.~ a windows PC.
Therefore the GUI setup files may use a different
working directory, containing: 
\bbul
\item Data files for startup panels (XML).
\item Configuration files for GUI (XML).
\ebul
These configuration files for the GUI are described in more detail 
in the GUI chapter below.
Of course both setups, for the \dabc\ application and the GUI, can be
put into one working directory if the GUI has access to it.

The \dabc\  setup files are application specific. Therefore they are described in
the following sections which cover typical example applications.

\subsection[DABC data simulator]{\dabc\ data simulator}
%\paref{user:Bnet}
\subsection[DABC MBS event server]{\dabc\ \mbs\ event server}
\paref{user:mbs}
\subsection{PCI connected front-ends}
%\paref{user:roc}


\lsection[Installation of additional plug-ins]{user-plugins}{Installation of additional plug-ins}
\index{DABC!Plug-in installation}
Apart from the \dabc\ base package, there may be additional plug-in packages for
specific use cases. Generally, these plug-in packages may consist of a
\strong{plugins} part and an \strong{applications} part.
The {\em plugins} part offers a library
containing new components (like \class{Devices}, 
\class{Transports}, or \class{Modules}). The {\em applications} part
mostly contains the XML setup files to use these new components in the
\dabc\ runtime environment; however, it 
may contain an additonal library defining the \dabc\ \class{Application}
class.

As an example, we may consider a plug-in package for reading out data
from specific PCIe hardware like the Active Buffer Board \ABB\ \cite{AbbDescription}.
This package is separately available for download at \hyperref{http://dabc.gsi.de}{}{}{http://dabc.gsi.de}
and described in detail in chapter \ref{prog-exa-pci-chapter} of the \dabc\ programmer's manual.

There are principally two different ways to install such separate plug-in packages:
Either within the general {\tt DABCSYS} directory as part of the central \dabc\ installation, as
described in following section \ref{user-plugins-dabcsys}. Or at an independent location
in a user directory, as described in section \ref{user-plugins-userdir}.


\lsubsection[Add plug-in packages to \$DABCSYS]{user-plugins-dabcsys}{Add plug-in packages to \$DABCSYS}
This is the recommended way to install a plug-in package if this package should be provided
for all users of the \dabc\ installation. A typical scenario would be that an
experimental group owns dedicated DAQ machines with system manager priviliges.
In this case, the plugin-package may be installed under the same account as the
central \dabc\ installation (probably, but not necessarily even the \keyw{root} account).
The new plug-in package should be directly installed in the {\tt \$DABCSYS} directory
then, with the following steps:

\bnum
\item Download the plug-in package tarball, e.~g.~ {\tt abb1.tar.gz}

\item Call the {\tt dabclogin.sh} script of the \dabc\ installation (see section user-env)

\item Copy the downloaded tarball to the {\tt \$DABCSYS} directory and unpack it there:
\begin{verbatim}
cp abb1.tar.gz $DABCSYS
cd $DABCSYS
tar zxvf abb1.tar.gz
\end{verbatim} 
This will extract the new components into the appropriate {\tt plugins} and
{\tt applications} folders below {\tt \$DABCSYS}.

\item Build the new components with the top Makefile of {\tt \$DABCSYS}:
\begin{verbatim}
make
\end{verbatim} 

\item To work with the new components, the configuration script(s) of the {\em applications} part should be copied
to the personal workspace of each user (see section \ref{user-setup}).
For the \ABB\ example, this is found at
\begin{verbatim}
$DABCSYS/applications/bnet-test/SetupBnetIB-ABB.xml
\end{verbatim} 
\enum





\lsubsection[Plug-in packages in user directory]{user-plugins-userdir}{Plug-in packages in user directory}
This is the case when \dabc\ is installed centrally at the fileserver
of an institute, and several experimental groups shall use different plug-ins.
It is also the recommended way if several users 
want to modify the source code of a plug-in library independently without 
affecting the general installation.


The new plug-in package should be installed in a user directory
then, with the following steps:

\bnum
\item Download the plug-in package tarball, e.~g.~ {\tt abb1.tar.gz}

\item Create a directory to contain your additional \dabc\ plugin packages:
\begin{verbatim}
mkdir $HOME/mydabcpackages
\end{verbatim} 

\item Call the {\tt dabclogin.sh} script of the \dabc\ installation (see section user-env)

\item Copy the downloaded tarball to the {\tt \$DABCSYS} directory and unpack it there:
\begin{verbatim}
cp abb1.tar.gz $HOME/mydabcpackages
cd $HOME/mydabcpackages
tar zxvf abb1.tar.gz
\end{verbatim} 
This will extract the new components into the appropriate {\tt plugins} and
{\tt applications} folders below the working directory. 

\item To build the {\em plugins} part, change to the appropriate package plugin
directory and invoke the local Makefile, e.~g.~ for the \ABB\ example:

\begin{verbatim}
cd $HOME/mydabcpackages/plugins/abb
make
\end{verbatim} 
This will create the corresponding plug-in library in a subfolder denoted by the
computer architecture, e.~g.~:
\begin{verbatim}
$HOME/mydabcpackages/plugins/abb/x86_64/lib/libDabcAbb.so
\end{verbatim} 


\item For some plug-ins, there may be also small test executables with different Makefiles in subfolder {\tt test}. These can be optionally build and executed independent of the
\dabc\ runtime environment.

\item The \dabc\ working directory for the new plug-in will be located in
subfolder 
\begin{verbatim}
applications/plugin-name 
\end{verbatim}
For the \ABB\ example, the application
will set up a builder network with optional Active Buffer Board readouts, so this
is at
\begin{verbatim}
$HOME/mydabcpackages/applications/bnet-test
\end{verbatim}
As in this example, there may be an additional library to be build containing the actual
\class{Application} class. This is done by invoking the Makefile within the directory:
\begin{verbatim}
cd $HOME/mydabcpackages/applications/bnet-test
make
\end{verbatim}
Here the application library is produced directly on top of the working directory: 
\begin{verbatim}
$HOME/mydabcpackages/applications/bnet-test/libBnetTest.so
\end{verbatim}

\item The actual locations of the newly build libraries (plugins, and optionally applications part) has to be edited in the 
\keyw{<lib>} tag of the corresponding \dabc\ setup-file (here: {\tt SetupBnetIB-ABB.xml}).
The default set-up examples in the plug-in packages assume that the library is located
at {\tt \$DABCSYS/lib}, as it is in the alternative installation case as described in
section \ref{user-plugins-dabcsys}.

\enum



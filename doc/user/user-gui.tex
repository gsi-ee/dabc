[user/user-gui.tex]
\section{GUI Guide lines}
\section{GUI Panels}
\figpng{user-gui-main-buttons}{Main buttons.}{htb}{0}{1.0}
Fig. \ref{fig:user-gui-main-buttons} shows the the main menu of \dabc\ (minimal view).\\
\icon{fileclose} Close\\
\icon{control} Test, shell script\\
\icon{savewin} Save settings\\
\icon{dabcmbsicon} \dabc\ \mbs\ launcher\\
\icon{dabcicon} \dabc\ launcher\\
\icon{mbsicon} \mbs\ launcher\\
\icon{browser}  Browser.\\
\icon{comicon} Command.\\
\icon{paramwin} Parameter table\\
\icon{usericon} Parameter selection panel\\
\icon{meterwin} Rate meter panel\\
\icon{histowin} Histogram panel\\
\icon{statewin} State panel\\
\icon{infowin} Info panel\\
\icon{logwin} Log panel\\
The three launcher panels (\dabc, \mbs, combined \dabc\ and \mbs) are used depending on the
application to be controlled. Eventually an application provides additional specific panels.
\subsection{\dabc~ launch panel}
\figpng{user-gui-pan-dabc}{DABC launcher.}{htb}{0}{0.6}
Values from file {\tt DabcLaunch.xml}
{\small \begin{verbatim}
<?xml version="1.0" encoding="utf-8"?>
<DabcLaunch>
<DabcMaster prompt="DABC Master" value="node.xxx.de" />
<DabcName prompt="DABC Name" value="Controller:41" />
<DabcUserPath prompt="DABC user path" value="myWorkDir" />
<DabcSystemPath prompt="DABC system path" value="/dabc" />
<DabcSetup prompt="DABC setup file" value="SetupDabc.xml" />
<DabcScript prompt="DABC Script" value="ps" />
<DabcServers prompt="%Number of needed DIM servers%" value="5" />
</DabcLaunch>
\end{verbatim}
\bdes
\item[DabcMaster:] Node where the master controller shall be started.
Can be one of the worker nodes.
\item[DabcName:] A unique name inside \dabc\ of the system.
\item[DabcUserPath:] User working directory. The GUI need not to have
access to the filesystem. 
\item[DabcSystemPath:] Path where the \dabc\ is installed. GUI needs no access to this path.
\item[DabcSetup:] Setup file name.
\item[DabcScript:] Command to be executed in an ssh at the master node.
\item[DabcServers:] Number of workers and controllers. This information
is minimum for the GUI to know when all \dabc\ nodes are up. The GUI waits until
this number of DIM servers is up and running.
\edes
\subsection{\mbs~ launch panel}
\figpng{user-gui-pan-mbs}{MBS launcher.}{htb}{0}{0.6}
Values from file {\tt MbsLaunch.xml}
{\small \begin{verbatim}
<?xml version="1.0" encoding="utf-8"?>
<MbsLaunch>
<MbsMaster prompt="MBS Master" value="node-xx" />
<MbsUserPath prompt="MBS User path" value="myMbsDir" />
<MbsSystemPath prompt="MBS system path" value="/mbs/v51" />
<MbsScript prompt="MBS Script" value="script/remote_exe.sc" />
<MbsCommand prompt="Script command" value="whatever command" />
<MbsServers prompt="%Number of needed DIM servers%" value="3" />
</MbsLaunch>
\end{verbatim}
\bdes
\item[MbsMaster]: Lynx node where the \mbs\ prompter is started.
\item[MbsUserPath]: User working directory. The GUI need not to have
access to the filesystem.
\item[MbsSystemPath]:  Path where the \mbs\ is installed. GUI needs no access to this path.
\item[MbsScript]: An execution script located relativ to \mbs path.
It sets all \mbs\ related variables and executes a program given as argument.
\item[MbsCommand]: Script command (argument to script).
\item[MbsServers]: Number of nodes plus prompter. This information
is minimum for the GUI to know when all \mbs\ nodes are up. The GUI waits until
this number of DIM servers is up and running.
\edes
\subsection{Combined \dabc~ and \mbs~ launch panel}
This panel is simply a superposition of the single ones.
\subsection{Command panel}
\figpng{user-gui-pan-cmd}{Command panel.}{htb}{0}{0.9}
On the left side is the command tree. Normally the tree is built
from name, application, nodes. Double click (or \keyw{RETURN}) on a treenode
executes the command on all treenodes below. The \mbs\ commands
are defined in a way that the top tree level is the executing \mbs\ task,
the next one the command, and the master node (prompter node) is the only node.
Command is sent to the current node which is displayed in the info panel
(see Fig. \ref{user-gui-pan-info}).
Click on a command opens at the right side the argument panel.
Entering argument values and \keyw{RETURN} executes the command.
\subsection{Monitoring panels}
\subsubsection{Rate meters}
\figpng{user-gui-pan-rate}{Rates.}{htb}{0}{0.7}
All rate meters are displayed in the meter panel. Meters can be removed
in the parameter table (See Fig. \ref{user-gui-pan-partab}) 
with the \keyw{Show} buttons like the other graphical parameters.
Saving the setup, the visibility will be preserved.
\figpdf{user-gui-rate-set}{Steering menus.}{htb}{0}{1}
On the left side in Fig. \ref{user-gui-rate-set} the \keyw{Settings}
menu is shown. It affects all items in the panel. One can \keyw{Zoom}
(toggle between large and normal view), change the number of
columns, change the display mode, toggle \keyw{Autoscale}, and set limits
(applied to all meters). Besides that each individual item can be
adjusted by right mouse button. The context menu is shown on the right.
All changes done individually are changing the defaults!
The global changes can be overwritten by these defaults.
All settings are saved with the setup.
\subsubsection{Histograms}
\subsubsection{States}
\figpng{user-gui-pan-state}{States.}{htb}{0}{1.0}
\subsubsection{Information}
\figpng{user-gui-pan-info}{Info.}{htb}{0}{0.7}
\subsection{Parameter table}
\figpng{user-gui-pan-partab}{Parameter table.}{htb}{0}{0.9}
\subsubsection{Parameter selection}
\figpng{user-gui-pan-parsel}{Parameter selection panel and selected parameter list.}{htb}{0}{1.0}
\subsection{Logging window}
\figpng{user-gui-pan-log}{Logging.}{htb}{0}{1.0}
\section{GUI save/restore setups}
There are several setups which can be stored in XML files and are retrieved
when the \gui\ is started again.
\bdes
\item [\keyw{DABC\_LAUNCH\_DABC}]: 
Values of \dabc\ launch panel. Saved by button in panel. \\
Default \verba{DabcLaunch.xml}. Filename in panel itself.
\item [\keyw{DABC\_LAUNCH\_MBS}]: 
Values of \mbs\ launch panel. Saved by button in panel. \\
Default \verba{MbsLaunch.xml}. Filename in panel itself.
\item [\keyw{DABC\_RECORD\_ATTRIBUTES}]: 
Attributes of records. Saved by main save button. \\
Default \verba{Records.xml}.
\item [\keyw{DABC\_PARAMETER\_FILTER}]: 
Values of parameter filter panel. Saved by main save button. \\
Default \verba{Selection.xml}.
\item [\keyw{DABC\_GUI\_LAYOUT}]: 
Layout of frames. Saved by main save button. \\
Default \verba{Layout.xml}.
\edes
\section{List of icons}
\icon{browser}  Browser.\\
\icon{comicon} Command.\\
\icon{conndsp} Connect single MBS\\
\icon{connprm} Connect MBS prompter\\
\icon{control} Test, shell script\\
\icon{dabcconfig} Configure\\
\icon{dabcicon} \dabc\ launcher\\
\icon{dabcmbsicon} \dabc\ \mbs\ launcher\\
\icon{dabcstart} Start acquisition\\
\icon{dabcstop} Hold acquisition\\
\icon{disconn} Shut down all \\
\icon{exitall} Exit all\\
\icon{fileclose} Close\\
\icon{filesave} Save\\
\icon{histowin} Histogram panel\\
\icon{info} Show acquisition\\
\icon{infowin} Info panel\\
\icon{logwin} Log panel\\
\icon{mbsconfig} Configure\\
\icon{mbsicon} \mbs\ launcher\\
\icon{mbsstart} Start acquisition\\
\icon{mbsstop} Stop acquisition\\
\icon{meterwin} Rate meter panel\\
\icon{paramwin} Parameter table\\
\icon{rshmbs} \mbs\ remote shell script\\
\icon{savewin} Save settings\\
\icon{statewin} State panel\\
\icon{usergraphics} Graphics panel\\
\icon{usericon} Parameter selection panel\\
\icon{user} Windows\\
\icon{windowclose} Windows close\\


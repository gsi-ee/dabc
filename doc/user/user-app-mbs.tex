[user/user-app-mbs.tex]
\label{user-app-mbs-chapter}
\lsection[MBS event building with DABC]{user:mbsapp}{\mbs\ event building with \dabc}
To run \mbs\ front-ends with \dabc\ nodes as event builders some
modifications of the \mbs\ setup files must be done.
For the \dabc\ side setup files must be provided.
\subsection[MBS setup]{\mbs\ setup}
When we want to use \dabc\ nodes as event builders, we need a different
setup on the \mbs\ side. We assume that we have more than one
\mbs\ node. Such multi-node system is controlled by an \mbs\ prompter running on one node.
The setup has to be changed such that all nodes run as if they are stand alone.
That means that each node runs the \keyw{Readout} - \keyw{Collector} - \keyw{Transport}
chain. The \dabc\ event builders connect to these transports.
The \mbs\ buffer size should be set to the stream size and the number of buffers per
stream must be set to one.

\subsection[DABC setup]{\dabc\ setup}
On the \dabc\ user working directory we need configuration files.

Summary of parameters:
\bdes
\item[MbsFileName] File name for list mode data file (\keyw{LMD}). Overwritten by command.
\item[MbsFileSizeLimit] File closes when size is reached, and new file opens.
\item[BufferSize] Should match \mbs\ buffer size.
\item[MbsServerKind] \keyw{Transport} | \keyw{Stream}.
\item[MbsServerPort] Port number transport (6000).
\item[MbsServerName] \mbs\ node of transport.
\item[NumInputs] Number of \mbs\ channels for one combiner.
\item[DoFile] Provide output file.
\item[DoServer] Provide server.
\edes
These parameters are used to configure an event generator:
\bdes
\item[NumSubevents]
\item[FirstProcId]
\item[SubeventSize]
\item[Go4Random]
\edes
\index{TODO!dabcsetupfiles}
The following example configuration file {\tt \$DABCSYS/applications/mbs/Combiner.xml} shows how to 
configure one combiner module reading from three \mbs\ transport servers.
Any running program is described by a \keyw{Context}, in this case
named \keyw{Combiner}. The \keyw{Run} specifies the
library, start function and log file name.
We have one module \keyw{Combiner} (could be different name from context) 
with three input ports and two output ports.

Now we can use the combined  controller panel to startup \mbs\ and \dabc.
\lsubsection[Combined DABC and MBS control panel]{user:controlDabcMbs}{Combined \dabc\ and \mbs\ control panel}
\figpng{user-gui-pan-dabcmbs}{Combined DABC and MBS  controller.}{htb}{0}{0.7}
This panel shown in Fig. \paref{fig:user-gui-pan-dabcmbs} is simply a superposition of the single ones.

\subsubsection[Combined DABC and MBS  controller buttons]{Combined \dabc\ and \mbs\  controller buttons}
\icon{savewin} Save panel settings, see \paref{user:guiSaveRestore}.\\
\icon{connprm}  Execute script \verba{dabcstartup.sc} at \dabc\ master node.
Starts DIM servers.
Execute script \verba{prmstartup.sc} at \mbs\ master node.
Starts prompter, dispatchers and message loggers.
Waits for all components.
A progress panel pops up during that time
(see \paref{user:guiProgress}).
If all components are up trigger the main \keyw{Update}.\\
\icon{dabcconfig} Configure. Execute user's \mbs\ startup procedure in prompter.
Executes state transition command \comm{Configure}
on \dabc\ master node and wait for the transition.
All plug-in components are created. Then execute \comm{Enable}.
If all components are up trigger the main \comm{Update}.\\
\icon{dabcstart} Start \mbs\ acquisition, then executes \dabc\ \comm{Start} command.
All components go into running state \keyw{Running}.\\
\icon{dabcstop} Pause acquisition. Execute \mbs\ \comm{stop acquisition}.
Execute \dabc\ \comm{Stop} command.
All components go into standby state \keyw{Ready}.\\
\icon{mbsstop} Halt acquisition. Executes \dabc\ \comm{Halt} command.
This closes all plug-ins. States go into \keyw{Halted}. 
Execute user's \mbs\ shutdown procedure in prompter.
Prompter, dispatcher and message loggers should still be running.
Next must be shut down or configure.
After two seconds trigger the main \keyw{Update}.\\
\icon{disconn} Shut down all. Execute \comm{EXIT} command on all \dabc\ nodes.
Execute script \verba{prmshutdown.sc} at \mbs\ master node.
After two seconds trigger the main \keyw{Update}.\\
\icon{info} \mbs\ \comm{Show acquisition}. Output in log panel.\\
\icon{controlmbs} Shell script for \mbs\ master node.\\
\icon{controldabc} Shell script for \dabc\ master node.


[user/user-overview.tex]
\lsection[Outline of this manual]{user-overview-outline}{Outline of this manual}
This \dabc\ User Manual contains all information that is necessary to
install and use the \dabc\ framework.

Chapter \paref{user-introduction} should be useful to understand the
most commonly used terms of \dabc.

Chapter \paref{user-setup-chapter} describes how to install the
\dabc\ packages on any linux machine, and how to set up the working
environment. Additionally, some typical use cases and their configuration
files are shown. 
The following chapters then give more detailed explanations how to operate
in different modes with the \dabc\ Java GUI:

Chapter \paref{user-gui-chapter}
covers the general functionality of the GUI
which is common for most applications. Especially, this 
is mostly sufficient
to control a DAQ cluster purely with one or several \dabc\ nodes.

Chapter \paref{user-gui-mbs-chapter} describes the \dabc\ GUI 
in a mode to control a pure \mbs data acquisition system without 
a native \dabc\ node.

The application use case for a mixed DAQ cluster, both with \dabc\ and \mbs\ nodes, is
treated in Chapter \paref{user-app-mbs-chapter}.

Finally, Chapter \paref{user-app-bnet-chapter} describes the use case of
a \dabc\ builder network (BNET), both with and without using \mbs\ .

However, the scope of the \dabc\ User Manual does not contain 
detailed descriptions of the \dabc\ framework architecture, 
the software mechanisms, and the example programs. 
These subjects are treated thouroughly
in the \dabc\ Programmer Manual.

\lsection[Release Notes]{user-overview-release}{Release Notes}

\subsection{Version 1.0.01 (10. March 2009)}
\bbul
\item Bugfix: suppress output of scripts running from ssh (caused problems with GUI).
\item Bugfix: GUI: Register DIM service after full instantiation of parameter object.
\item Bugfix: GUI: Histogram drawer had uninitialized field.
\ebul
\subsection{Version 1.0.00 (26. February 2009)}
These are the features of the first official release:
\bnum
\item A Data Acquisition framework in C++ language for linux platforms
   with modular components for dataflow on multiple nodes.
   
\item Runtime environment with basic services for:
   threads, event handling, memory management, command execution, 
   configuration, logging, error handling

\item Plug-in mechanism for user defined DAQ applications

\item Plug-in mechanism for a control system. Features a finite state machine
   logic and parameters for monitoring and  configuration.
   The default implementation is based
   on the DIM protocol (http://dim.web.cern.ch/dim)

\item Java GUI to operate the standard DIM control system of DABC/MBS. 
   Fully generic evaluating DABC process variables, but extendable
   by user written components.

\item Contains a sub-framework to set-up distributed event builder networks (BNET) 
 
\item Supports tcp/ip and InfiniBand/verbs networks for data transport

\item Supports formats and readout of GSI's standard DAQ system MBS
   (Multi Branch System). May also write data into MBS listmode format,
   and may emulate MBS socket data servers.
   Additionally, MBS systems can be controlled by the DABC GUI.  
\enum

% 
% More specific functionality is described in the sections of the applications
% (as far as delivered with \dabc).
% 
% As examples we take sometimes the \dabc\ simulator plug-in which runs out of the box.
% 
% External application specific GUI add-ons cannot be described here.


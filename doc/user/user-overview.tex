[user/user-overview.tex]

\lsection[About DABC]{user-overview-about}{About \dabc\ }
%Here general philosophy, mabe take part of introduction manual
The Data Acquisition Backbone Core \dabc\ is a
Data Acquisition (DAQ) framework with modular components for dataflow on multiple nodes.
It provides a C++ runtime environment with all basic services, such as:
threads and event handling; memory management; command execution; 
configuration; logging; and error handling.
User written DAQ applications can be run within this environment by
means of a plug-in mechanism.

\dabc\ contains the \strong{BNET} subframework with additional interfaces 
to set-up distributed event builder networks. As transport layers for such
networks, {\em tcp/ip} and {\em InfiniBand/verbs} are supported.

\dabc\ supports by default the data formats and readout connections of GSI's standard DAQ system \mbs\ (Multi Branch System). It may also write datafiles with the 
\mbs\ {\tt *.lmd} format, and it may emulate \mbs\ data server sockets, such as
{\em stream} or {\em transport} servers.

The \dabc\ control system features a finite state machine
logic and parameters for monitoring and  configuration. This is also
designed as a plug-in to the base framework. The current implementation is based
on the DIM protocol \cite{DIM}.
A generic Java GUI is provided to operate this standard DIM control system.
This GUI may also control \mbs\ systems which support the DIM communication.
It is extendable by user written components. 




\lsection[Outline of this manual]{user-overview-outline}{Outline of this manual}
This \dabc\ User Manual contains all information that is necessary to
install and use the \dabc\ framework.

Chapter~\ref{user-setup-chapter} describes how to install the
\dabc\ packages on any linux machine, and how to set up the working
environment. Additionally, some typical use cases and their configuration
files are shown. 
The following chapters then give more detailed explanations how to operate
in different modes with the \dabc\ Java GUI:

Chapter~\ref{user-gui-chapter}
covers the general functionality of the GUI
which is common for most applications. Especially, this 
is mostly sufficient
to control a DAQ cluster purely with one or several \dabc\ nodes.

Chapter~\ref{user-gui-mbs-chapter} describes the \dabc\ GUI 
in a mode to control a pure \mbs data acquisition system without 
a native \dabc\ node.

Finally, the GUI mode for a mixed DAQ cluster, both with \dabc\ and \mbs\ nodes, is
treated in Chapter~\ref{user-app-mbs-chapter}.

However, the scope of the \dabc\ User Manual does not contain 
detailed descriptions of the \dabc\ framework architecture, 
the software mechanisms, and the example programs. 
These subjects are treated thouroughly
in the \dabc\ Programmer Manual.

\lsection[Release Notes]{user-overview-release}{Release Notes}

\subsection{Version 1.0.0 (March 2009)}
These are the features of the first official release:
\bnum
\item A Data Acquisition framework in C++ language for linux platforms
   with modular components for dataflow on multiple nodes.
   
\item Runtime environment with basic services for:
   threads, event handling, memory management, command execution, 
   configuration, logging, error handling

\item Plug-in mechanism for user defined DAQ applications

\item Plug-in mechanism for a control system. Features a finite state machine
   logic and parameters for monitoring and  configuration.
   The default implementation is based
   on the DIM protocol (http://dim.web.cern.ch/dim)

\item Java GUI to operate the standard DIM control system of DABC/MBS. 
   Fully generic evaluating DABC process variables, but extendable
   by user written components.

\item Contains a sub-framework to set-up distributed event builder networks (BNET) 
 
\item Supports tcp/ip and InfiniBand/verbs networks for data transport

\item Supports formats and readout of GSI's standard DAQ system MBS
   (Multi Branch System). May also write data into MBS listmode format,
   and may emulate MBS socket data servers.
   Additionally, MBS systems can be controlled by the DABC GUI.  
\enum

% 
% More specific functionality is described in the sections of the applications
% (as far as delivered with \dabc).
% 
% As examples we take sometimes the \dabc\ simulator plug-in which runs out of the box.
% 
% External application specific GUI add-ons cannot be described here.


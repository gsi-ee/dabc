[user/user-exa-mbs.tex]
\section{Example \mbs\ event building}
\subsection{\mbs\ GUI}
The \dabc\ GUI can be used to control a stand alone \mbs\ system
or a combined \mbs\ (front-ends) \dabc\ (event builder) system.
To control a standard \mbs\ nothing has to be done by the user on the \mbs\ side.
Of cause, the \mbs\ itself must have been build with the DIM option (since version v5.1).
The \mbs\ runs like with the prompter. Central log file is written as usual.
\lsubsection{user:launchMbs}{\mbs\ launch panel}
\figpng{user-gui-pan-mbs}{MBS launcher.}{htb}{0}{0.6}
Values from file {\tt MbsLaunch.xml} (default, may be saved to other file,
see \paref{user:guiSaveRestore}).
{\small \begin{verbatim}
<?xml version="1.0" encoding="utf-8"?>
<MbsLaunch>
<MbsMaster prompt="MBS Master" value="node-xx" />
<MbsUserPath prompt="MBS User path" value="myMbsDir" />
<MbsSystemPath prompt="MBS system path" value="/mbs/v51" />
<MbsScript prompt="MBS Script" value="script/remote_exe.sc" />
<MbsCommand prompt="Script command" value="whatever command" />
<MbsServers prompt="%Number of needed DIM servers%" value="3" />
</MbsLaunch>
\end{verbatim}
\bdes
\item[MbsMaster]: Lynx node where the \mbs\ prompter is started.
\item[MbsUserPath]: User working directory. The GUI need not to have
access to the filesystem.
\item[MbsSystemPath]:  Path where the \mbs\ is installed. GUI needs no access to this path.
\item[MbsScript]: An execution script located relativ to \mbs\ path.
It sets all \mbs\ related variables and executes a program given as argument.
\item[MbsCommand]: Script command (argument to MbsScript).
\item[MbsServers]: Number of nodes plus prompter. This information
is minimum for the GUI to know when all \mbs\ nodes are up. The GUI waits until
this number of DIM servers is up and running.
\edes
That file can be created from within the GUI in the \mbs\ launcher panel.
Enter all values necessary, and store them. 
\subsubsection{\mbs\ launcher buttons}
\icon{fileclose} Close window.\\
\icon{savewin} Save panel settings, see \paref{user:guiSaveRestore}.\\
\icon{connprm}  Execute script \verba{prmstartup.sc} at master node.
Starts prompter, dispatchers and message loggers and waits untill they are up.
Trigger the main \keyw{Update}.
A progress panel pops up during that time (see \paref{user:guiProgress}).\\
\icon{conndsp} Execute script \verba{dimstartup.sc} at master node.
Starts dispatcher and message logger for single node \mbs.
Trigger the main \keyw{Update}.\\
\icon{dabcconfig} Configure. Execute \verba{startup.scom} in prompter (dispatcher).
Trigger the main \keyw{Update}.\\
\icon{dabcstart} \comm{Start acquisition}.\\
\icon{dabcstop} Hold acquisition. Execute \comm{stop acquisition}.\\
\icon{mbsstop} Stop acquisition. Execute \verba{shutdown.scom} in prompter.
Prompter, dispatcher and message loggers should still be running.\\
\icon{disconn} Shut down all. Execute script \verba{prmshutdown.sc} at master node.
After 2 seconds trigger the main \keyw{Update}.\\
\icon{info} \comm{Show acquisition}. Output in log panel.\\
\icon{rshmbs} Executes \keyw{Script} with argument \keyw{Command} at master node.\\
\icon{controlmbs} Shell script

\lsubsection{user:launchDabcMbs}{Combined \dabc\ and \mbs\ launch panel}
\figpng{user-gui-pan-dabcmbs}{Combined DABC and MBS launcher.}{htb}{0}{0.7}
This panel shown in Fig. \pageref{user-gui-pan-dabcmbs} is simply a superposition of the single ones.
\subsubsection{Combined \dabc\ and \mbs\ launcher buttons}
\icon{fileclose} Close window.\\
\icon{savewin} Save panel settings, see \paref{user:guiSaveRestore}.\\
\icon{connprm}  Execute script \verba{dabcstartup.sc} at \dabc\ master node.
Starts DIM servers.
Execute script \verba{prmstartup.sc} at \mbs\ master node.
Starts prompter, dispatchers and message loggers.
Waits for all components.
A progress panel pops up during that time
(see \paref{user:guiProgress}).
If all components are up trigger the main \keyw{Update}.\\
\icon{dabcconfig} Configure. Execute \verba{startup.scom} in \mbs\ prompter.
Executes state transition command \comm{Configure}
on \dabc\ master node and wait for the transition.
All plug-in components are created. Then execute \comm{Enable}.
If all components are up trigger the main \comm{Update}.\\
\icon{dabcstart} Start \mbs\ acquisition, then executes \dabc\ \comm{Start} command.
All components go into running state \keyw{Running}.\\
\icon{dabcstop} Hold acquisition. Execute \mbs\ \comm{stop acquisition}.
Execute \dabc\ \comm{Stop} command.
All components go into standby state \keyw{Ready}.\\
\icon{mbsstop} Executes \dabc\ \comm{Halt} command.
This closes all plug-ins. States go into \keyw{Halted}. 
Execute \verba{shutdown.scom} in \mbs\ prompter.
Prompter, dispatcher and message loggers should still be running.
Next must be shut down or configure.
After two seconds trigger the main \keyw{Update}.\\
\icon{disconn} Shut down all. Execute \comm{EXIT} command on all \dabc\ nodes.
Execute script \verba{prmshutdown.sc} at \mbs\ master node.
After two seconds trigger the main \keyw{Update}.\\
\icon{info} \comm{Show acquisition}. Output in log panel.\\
\icon{rshmbs} Executes \keyw{Script} with argument \keyw{Command} at master node.\\
\icon{controlmbs} Shell script for \mbs\ master node.\\
\icon{controldabc} Shell script for \dabc\ master node.
\subsection{\mbs\ command panel}
\figpng{user-gui-pan-cmd-mbs}{Command panel.}{htb}{0}{0.9}
Fig. \paref{fig:user-gui-pan-cmd-mbs} shows
on the left side the command tree. Double click (or \keyw{RETURN}) on a command
executes the command. The top tree level is the executing \mbs\ task,
below that are the commands, and the master node (prompter node) is the only node
below each command. However,
command is sent to the prompter node, but executed on the current node 
which is displayed in the info panel
(see Fig. \paref{fig:user-gui-pan-info}).
Click on a command opens at the right side the argument panel.
Entering argument values and \keyw{RETURN} executes the command.
\section{\mbs\ DIM parameters}
\subsection{\mbs\ states}
\bdes
\item[Acquisition/State] \keyw{Running} | \keyw{Stopped} 
\item[BuildingMode/State] \keyw{Delayed} | \keyw{Immediate}
\item[EventBuilding/State] \keyw{Working} | \keyw{Suspended}
\item[FileOpen/State] \keyw{File open} | \keyw{File closed}
\item[RunMode/State] \keyw{DABC connected} | \keyw{MBS to DABC} | \keyw{Transport client} | \keyw{MBS standalone}
\item[SpillOn/State] \keyw{Spill ON} | \keyw{Spill OFF}
\edes
\subsection{\mbs\ rates}
\bdes
\item[MSG/DataRateKb] KByte/s
\item[MSG/DataTrendKb] KBytes/s as trend
\item[MSG/EventRate] Events/s
\item[MSG/EventTrend] Events/s as trend
\item[MSG/StreamRateKb] Stream server Kbyte/s
\item[MSG/StreamTrendKb] Stream server Kbyte/s as trend
\item[MSG/FileFilled] File filled in percent
\item[MSG/StreamsFull] Number of full streams in percent
\edes
\subsection{\mbs\ infos}
Shown in info window.
\bdes
\item[MSG/eFile] Name of file.
\item[MSG/ePerform] Events, MBytes, Events/s and MBytes/s.
\item[MSG/eSetup] Name of setup file loaded.
\item[PRM/Current] Current command execution node (master node only).
\item[PRM/NodeList] List of nodes (master node only).
\edes
\subsection{\mbs\ tasks}
When a task is started a parameter with it's name is created.
The value is the task name. If the task is stopped,
color changes to gray and value is preceded by "no".
Task list is shown in info window.
\bdes
\item[MSG/Dispatch]
\item[MSG/MsgLog]
\item[MSG/ReadMeb]
\item[MSG/Collector]
\item[MSG/Transport]
\item[MSG/EventServ]
\item[MSG/Util]
\item[MSG/ReadCam]
\item[MSG/EsoneServ]
\item[MSG/StreamServ]
\item[MSG/Histogram]
\item[MSG/Prompt]
\item[MSG/Rate]
\item[MSG/SMI]
\item[MSG/Sender]
\item[MSG/Receiver]
\item[MSG/AsynchReceiver]
\item[MSG/Rising]
\item[MSG/TimeOrder]
\item[MSG/VmeServ]
\edes
\subsection{\mbs\ text}
\bdes
\item[MSG/GuiNode] Node where GUI runs
\item[MSG/Date] Date as written in file header
\item[MSG/Run] Run ID  as written in file header
\item[MSG/Experiment] Experiment as written in file header
\item[MSG/User] Lynx user name as written in file header
\edes
\subsection{\mbs\ numbers}
\bdes
\item[MSG/BufferSize]
\item[MSG/Buffers] collected so far.
\item[MSG/Events] collected so far.
\item[MSG/FileMbytes] written in file.
\item[MSG/FlushTime]
\item[MSG/MBytes] collected so far.
\item[MSG/StreamKeep] 
\item[MSG/StreamMbytes]
\item[MSG/StreamScale]
\item[MSG/StreamSync]
\edes
\section{Working directories}
\subsection{\mbs\ configuration of DIM}
Optional file \verba{.guirc} in the \mbs\ working directory
specifies which rate meters and states shall appear
in the GUI. Upper limits of the rate meters can be specified.
This file can be copied from \$MBSROOT/set. Only the
parameters which are in this file are optional.
{\small {\begin{verbatim}
## This file controls the rate meter and state appearance.
## File name must be .guirc and in the MBS working directory.
## The value numbers are the maximum values for rate meters
## Colons only if value is specified!
## Node names must be uppercase, * wildcards all
## RunMode shows if MBS is set into DABC event builder mode 
## and client is connected.

##========= All nodes:
##---- Rates:
* EventRate     : 10000.
* EventTrend    : 10000.
* DataRateKb    : 16000.
* DataTrendKb   : 16000.
* StreamRateKb  : 16000.
* StreamTrendKb : 16000.
# ++ File filling status in percent, typically only on one node (transport)
#* FileFilled   :   100.
* StreamsFull   :   100.

##---- States:
# ++ Delayed or immediate event building:
* BuildingMode
# ++ Current eventbuilding running or suspended:
* EventBuilding
# ++ Running mode, stand alone or DABC:
* RunMode
# ++ Shows spill signal:
#* SpillOn
# ++ Shows if file is open, typically only on one node (transport)
#* FileOpen

##======== Node XXX
#XXX EventRate   : 10000.
#XXX DataRateKb  : 16000.
#XXX RunMode
#XXX FileOpen
#XXX FileFilled  :   100.
#XXX SpillOn
#XXX EventTrend  : 10000.
#XXX DataTrendKb : 16000.
\end{verbatim}
}
\subsection{\dabc\ configuration}


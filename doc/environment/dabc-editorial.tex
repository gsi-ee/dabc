\Chapter{Editorial}
[environment/dabc-editorial.tex]

\section{Structure of document}
The document is structured hierarchically. To make sure that files
to be included by \verb+\+input\{filename\} or \verb+\+include\{filename\}
can be located, set the following environment variables:\\
Linux:\\
export TEXINPUTS=\verb+<+topdirectory\verb+>+//:\\
Windows: If one uses fpTeX with WInEdt:\\
Append ;P:\verb+\+Application\verb+\+TeXLive2005\verb+\+bin\verb+\+win32 to PATH.\\
Set TEXINPUTS to x:\verb+\+topdirectory\verb+\+//;\\
(Systemsteuerung->System:Erweitert:Umgebungsvariablen)

The full document is built by command (we are on topdirectory):
\begin{verbatim}
pdflatex main-all
makeindex main-all.idx
pdflatex main-all
\end{verbatim}
or by {\tt make}. It builds the document in parts from the steering files in the directories.
On each subdirectory xxx there might be a main file main-xxx.tex to build a document 
from this directory only, i.e.
\begin{verbatim}
cd template
pdflatex main-template
makeindex main-template.idx
pdflatex main-template
\end{verbatim}
Alternatively on top directory the script 
\begin{verbatim}
makedoc <subdirectory>
\end{verbatim}
can be used. 
The files on directory {\tt template} can be used as templates, i.e. copied to a new subdirectory.
All occurences of XXX in file names and tex files should then be renamed properly.
The script {\tt rename.sh} can be used to do so:
\begin{verbatim}
. ../rename.sh XXX yyy
\end{verbatim}
replaces all {\tt XXX} to {\tt yyy} in tex file names and tex files.
(After that all eventually remaining {\tt *XXX*} files can be deletetd).\\
The file {\tt XXX-section.tex} contains commonly used tex commands.
It could be used as cut\&paste source.\\
Figures (pdf) can be located in any subdirectories, typically {\tt figures}.\\

\clearpage
Description of the files:
\subsubsection{Topdirectory}
\begin{compactdesc}
\item[Makefile] make file.
\item[makedoc] script to make a subdirectory.
\item[main-all.tex] main file to be texed. Includes all steer files from subdirectories.
\item[bibitem.tex] references
\item[dabc-glossary.tex] glossary
\item[dabc-requirements.tex] brief and informal list of requirements
\item[dabcclass.cls] document description
\item[rename.sh] script to rename/replace strings in file names and content.
\end{compactdesc}
\subsubsection{Subdirectory {\tt environment}}
\begin{compactdesc}
\item[dabc-docrev.tex] document name and revision information
\item[dabc-defs.tex] central definitions (included by all main files)
\item[dabc-post.tex] reference and index chapters (included by all main files)
\item[dabc-frontpage.tex] first page of top document
\item[dabc-people.tex] list of people
\item[dabc-preface.tex] this text
\item[dabc-work.tex] working packages
\end{compactdesc}
\subsubsection{Subdirectory {\tt controls}}
Example of a manual part. 
\begin{compactdesc}
\item[ctrl-docrev.tex] document name and revision information.
Is included by {\tt main-all.tex} and {\tt main-controls.tex}.
\item[main-controls.tex] main file to be texed.
Includes {\tt steer-controls.tex} and {\tt ctrl-docrev.tex}. Adjust document information here.
\item[steer-controls.tex] includes everything needed from this directory.
Is included by {\tt main-all.tex} and {\tt main-controls.tex}. Controls chapters.
\item[ctrl-section.tex] is an example section file (several sections)
 as included in the steering file.
\end{compactdesc}
All other directories below topdirectory have the main, docrev and the steer file.
\section{Formatting shortcuts}
Some macros are defined in the style file {\tt dabcclass.cls}
\subsection{Font styles}
\begin{compactitem}[$\bullet$] 
\item macro {\tt $\backslash$verba\{Verbatim\}} , tt \verba{Verbatim}
\item macro {\tt $\backslash$decl\{Declaration\}} , tt \decl{Declaration}
\item macro {\tt $\backslash$class\{Class\}} , bf em \class{Class}
\item macro {\tt $\backslash$func\{Function\}} , sl \func{Function}
\item macro {\tt $\backslash$member\{Member\}} , sl \func{Member}
\item macro {\tt $\backslash$strong\{Strong\}} , bf \strong{Strong}
\item macro {\tt $\backslash$keyw\{Keyword\}} , sf \keyw{Keyword}
\item macro {\tt $\backslash$param\{Parameter\}} , sf \param{Parameter}
\item macro {\tt $\backslash$comm\{Command\}} , sf \comm{Command}
\end{compactitem}
Example text:\\
When we have a \class{MyNewClass} it might have some \func{Functions} and some \member{Members}.
It also might have some \keyw{Constants} and \decl{Declarations}.
Fixed terms should be in \verba{Typewriter}. Text to be highlighted: \strong{Note!}.
DIM parameter and commands as \param{DataRate} and \comm{setBufferSize}
\subsection{Lists}
b is for begin, e for end
\begin{verbatim}
 {\bbul} = {\begin{compactitem}[$\bullet$]}
 {\ebul} = {\end{compactitem}}
 {\bcir} = {\begin{compactitem}[$\circ$]}
 {\ecir} = {\end{compactitem}}
 {\btri} = {\begin{compactitem}[$\triangleright$]}
 {\etri} = {\end{compactitem}}
 {\bbox} = {\begin{compactitem}[$\Box$]}
 {\ebox} = {\end{compactitem}}
 {\bnum} = {\begin{compactenum}}
 {\enum} = {\end{compactenum}}
 {\bdes} = {\begin{compactdesc}}
 {\edes} = {\end{compactdesc}}
\end{verbatim}
\bbul
\item bbul - ebul
\ebul
\bcir
\item bcir - ecir
\ecir
\btri
\item btri - etri
\etri
\bbox
\item bbox - ebox
\ebox
\bnum
\item bnum - enum
\enum
\bdes
\item[item] bdes - edes
\edes
\section{Naming conventions}

\Chapter{Preface}

This document describes the requirements, design, and implementation of the
general purpose data acquisition dabcnstrator system.
This system, called {\DDA}, is a result of the discussions about DAQ concepts for CBM,
Panda, and FutureDAQ started in 2004.

\section{Structure of document}
The document is structured hierarchically. To make sure that files
to be included by \verb+\+input\{filename\} or \verb+\+include\{filename\}
can be located, set the following environment variables:\\
Linux:\\
export TEXINPUTS=\verb+<+topdirectory\verb+>+//:\\
Windows: If one uses fpTeX with WInEdt:\\
Append ;P:\verb+\+Application\verb+\+TeXLive2005\verb+\+bin\verb+\+win32 to PATH.\\
Set TEXINPUTS to x:\verb+\+topdirectory\verb+\+//;\\
(Systemsteuerung->System:Erweitert:Umgebungsvariablen)

The full document is built by command (we are on topdirectory):
\begin{verbatim}
pdflatex main-all
makeindex main-all.idx
pdflatex main-all
\end{verbatim}
On each subdirectory there is a main file main-xxx.tex, i.e.
\begin{verbatim}
cd template
pdflatex main-XXX
makeindex main-XXX.idx
pdflatex main-XXX
\end{verbatim}
The files on directory {\tt template} can be used as templates, i.e. copied to a new subdirectory.
All occurences of XXX in file names and tex files should then be renamed properly.
The script {\tt rename.sh} can be used to do so:
\begin{verbatim}
. ./rename.sh XXX yyy
\end{verbatim}
replaces all {\tt XXX} to {\tt yyy} in tex file names and tex files.
(After that all {\tt *XXX*} files can be deletetd).\\
The file {\tt XXX-section.tex} contains commonly used tex commands.
It could be used as cut\&paste source.\\
Description of the files:
\subsubsection{Topdirectory}
\begin{compactdesc}
\item[main-all.tex] main file to be texed. Includes all steer files from subdirectories.
\item[bibitem.tex] references
\item[dabc-glossary.tex] glossary
\item[dabc-requirements.tex] brief and informal list of requirements
\item[dabcclass.cls] document description
\end{compactdesc}
\subsubsection{Subdirectory {\tt environment}}
\begin{compactdesc}
\item[dabc-docrev.tex] document name and revision information
\item[dabc-defs.tex] central definitions (included by all main files)
\item[dabc-post.tex] reference and index chapters (included by all main files)
\item[dabc-frontpage.tex] first page of top document
\item[dabc-people.tex] list of people
\item[dabc-preface.tex] this text
\item[dabc-work.tex] working packages
\end{compactdesc}
\subsubsection{Subdirectory {\tt controls}}
Example of a manual component. 
\begin{compactdesc}
\item[ctrl-docrev.tex] document name and revision information.
Is included by {\tt main-all.tex} and {\tt main-controls.tex}.
\item[main-controls.tex] main file to be texed.
Includes {\tt steer-controls.tex} and {\tt ctrl-docrev.tex}. Adjust document information here.
\item[steer-controls.tex] includes everything needed from this directory.
Is included by {\tt main-all.tex} and {\tt main-controls.tex}.
\end{compactdesc}
All other directories below topdirectory have the main, docrev and the steer file.
\section{Naming conventions}

\Chapter{Preface}
[environment/dabc-preface.tex]

This document describes the requirements, design, and implementation of the
general purpose data acquisition backbone core \dabc.
This system is a result of the discussions about DAQ concepts for CBM,
Panda, and FutureDAQ started in 2004.

There is a set of manuals which are available as single books, or
combined in one book with separate parts for each manual.

The manuals are:
\begin{compactdesc}
\item [Introduction and Overview] Gives a first idea about \dabc~, 
why it was developed, and what it is good for, the use cases and the requirements.
\item [User Manual] The basics how to use \dabc~. This manual cannot be complete,
because the \dabc~ as a backbone needs some application specific components,
which cannot described here, including application specific GUIs.
There are, however, some applications provided within the \dabc~ distribution
which are described, like the \mbs~ or \ROC~.
\item [Programmer Manual] The application specific components mentioned above
have to be implemented as plug-ins into the \dabc~ framework. These mechanisms
are described here.
\item [Reference Manual] Reference of all classes, interfaces and functions.
\item [Controls] Because the \dabc~ is divided in a core part and a controls environment part,
the current controls part might be replaced. This manual describes the current controls
as well as the implementation rules of another one.
\item [Java GUI Reference] As mentioned above, applications may need their specific
GUIs. Java written application GUIs can be hooked into the \dabc~ GUI. 
This is described in the programmers manual. The references can be found here.
\end{compactdesc}
